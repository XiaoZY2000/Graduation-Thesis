% ファイル名をmain.texに変更するとコンパイルエラーが発生するので注意
% 总计页数: 50
% Related Work: 10页
% Proposed Method: 15页

\documentclass[a4paper]{master_thesis}
\usepackage[dvipdfmx]{graphicx}
% \usepackage{graphicx}
\usepackage[dvipdfmx]{color}
% \usepackage[notransparent]{svg}

\usepackage{fancyhdr}
\usepackage{amsmath}
\usepackage{amssymb}
\usepackage{cleveref}

\usepackage{url}
\usepackage{booktabs}
\usepackage{siunitx}
\usepackage{here}
\usepackage{amsfonts}
\usepackage{bm}
\usepackage{mathtools}
\usepackage{comment}
\usepackage{mynewcommand}
% Algorithm
\usepackage{algorithm}
\usepackage{algpseudocode}
\usepackage{enumitem}
\usepackage{graphicx}
\usepackage{subcaption}

\addtolength{\topmargin}{-10pt}
\setlength{\jot}{10pt}

\bibliographystyle{junsrt}
\makeatletter 
\def\delete#1from#2{% 
  \def\reserved@a{#1}% 
  \def\reserved@c{}% 
  \def\@elt##1{% 
    \def\reserved@b{##1}% 
    \ifx\reserved@a\reserved@b 
    \else 
      \@cons\reserved@c{{##1}}% 
    \fi}% 
  \csname cl@#2\endcsname 
  \expandafter\let\csname cl@#2\endcsname\reserved@c 
  \let\@elt\relax} 
\makeatother
\delete{footnote}from{chapter}

\begin{document}
\title{Distributional User Preference Modeling for Cross-Domain Recommendation under Non-Overlapping Settings}
\subtitle{\LARGE 非重複設定下におけるクロスドメイン推薦のための分布的ユーザ嗜好モデリング}
\author{48-246453 肖 子吟}
\faculty{東京大学大学院 情報理工学系研究科}
\department{電子情報学専攻}
\professor{鈴村 豊太郎 教授}
\date{2026 年 1 月 22 日}

\maketitle
\onecolumn

\begin{abstract}
% What is recommendation systems?
% Why are they important?
% What are the challenges?
% Why use cross-domain recommendation?
% What is the problem with existing methods?
% What is the proposed method?
% What are the results?

Recommendation systems play a crucial role in helping users discover relevant
items across a wide range of applications, including e-commerce, streaming
services, and social media platforms.
With the rapid expansion of online platforms and available content, developing
effective recommendation systems has become increasingly important.

Despite their success, traditional recommendation systems often suffer from
data sparsity and cold-start problems, which significantly limit their
performance.
Cross-domain recommendation has therefore emerged as a promising paradigm that
leverages information from multiple domains to enhance recommendation accuracy.
However, most existing cross-domain recommendation methods rely on the
assumption of overlapping users or items across domains, an assumption that is
frequently violated in real-world scenarios.
Moreover, user preferences are typically represented as point estimates, which
are insufficient for capturing the uncertainty and multi-faceted nature of user
behavior.

In this thesis, we propose a novel cross-domain recommendation framework that
models user preferences as Gaussian Mixture Models (GMMs) and integrates optimal
transport to enable knowledge transfer under non-overlapping settings.
By representing user preferences as probability distributions, the proposed
method effectively captures the diversity and uncertainty inherent in user
interests.
Furthermore, optimal transport provides a principled mechanism for aligning
preference distributions across domains, allowing effective knowledge transfer
even in the absence of overlapping users or items.

Extensive experiments on multiple Amazon datasets demonstrate that the proposed
approach consistently outperforms state-of-the-art cross-domain recommendation
methods as well as strong single-domain baselines.
Additional ablation studies further verify the importance of distributional user
preference modeling and the effectiveness of incorporating source-domain
information to enhance target-domain recommendation performance.
\end{abstract}

\begin{comment}

\end{comment}

\clearpage

%\maketoc
\setcounter{page}{0}

\pagestyle{fancyplain}
\pagenumbering{roman}
\tableofcontents
% comment out if 
\listoffigures
\listoftables
%\listofalgorithms
\newpage

%\baselineskip 5.8mm
\pagestyle{fancy}
\lhead{\rightmark}
\renewcommand{\chaptermark}[1]{\markboth{Chapter\ \normalfont\thechapter\ \ ~#1}{}}
\renewcommand{\sectionmark}[1]{\markright{\thesection\ #1}{}}
\rhead{\leftmark}
\pagenumbering{arabic}



\chapter{Introduction}\label{chap:introduction}

\section{Background}

Information overload has become a critical challenge for users on modern online
platforms, including e-commerce websites, streaming services, and social
media~\cite{Bawden_Robinson_2020_info_overload,Lv_Liu_2022_info_overload,Zhang_Cao_Liu_2023_info_overload}.
Recommender systems play a crucial role in alleviating this problem by providing
personalized guidance, selecting items with the highest predicted utility from a
large pool of candidates~\cite{Resnick_Varian_1997_recommender,Burke_2002_hybrid_recommender,Burke_Felfernig_Göker_2011_recommender_overview}.
By analyzing historical user-item interactions, recommender systems help users
discover relevant content, products, or services, thereby improving user
satisfaction and engagement.
Beyond enhancing user experience, recommender systems also generate substantial
business value by increasing sales, customer retention, and overall platform
activity~\cite{Jannach_Jugovac_2019_business_value_recommender}.
As the scale and diversity of online content continue to grow, the importance of
effective recommender systems becomes increasingly pronounced.

Despite their success, recommender systems face several fundamental challenges
that limit their performance.
One major challenge is data sparsity, where users interact with only a small
fraction of available items, making it difficult to accurately infer user
preferences.
This issue is particularly severe in emerging or small-scale platforms with limited
interaction data.
Another critical challenge is the cold-start problem, which arises when new users
or items enter the system and insufficient historical data is available to
support reliable recommendations.
Both data sparsity and cold-start problems arise from the limited availability of
user-item interaction data and can significantly degrade recommendation quality.

To address these challenges, cross-domain recommendation has emerged as a
promising paradigm that leverages information from a related source domain to
improve recommendation performance in a target domain.
By transferring knowledge across domains, cross-domain recommendation is able to
alleviate data sparsity and cold-start issues, particularly in data-scarce
settings.
However, most existing cross-domain recommendation methods rely on the
assumption that users or items overlap between the source and target domains, and
such overlaps are used as bridges for knowledge transfer during the training
stage.
In many real-world scenarios, however, this assumption does not hold, as domains
may be entirely disjoint due to privacy constraints, system isolation, or
independent platform design.
Moreover, traditional recommender systems typically represent user preferences as
point estimates in a latent space, which are often insufficient to capture the
uncertainty and multi-faceted nature of real user behavior.

Motivated by these limitations, this thesis proposes a novel cross-domain
recommendation framework that models user preferences as Gaussian Mixture Models
(GMMs) and integrates optimal transport techniques to enable knowledge transfer
under non-overlapping training settings.
By representing user preferences as probability distributions rather than point
estimates, the proposed approach captures both the diversity and uncertainty of
user interests.
Furthermore, optimal transport provides a principled mechanism for aligning user
preference distributions across domains, facilitating effective knowledge
transfer without relying on overlapping users or items during the training
phase.

The proposed framework consists of two key components:
(1) modeling user preferences as Gaussian mixture distributions to capture the
complex structure of user behavior within each domain, and
(2) employing optimal transport to align these distributions between the source
and target domains.
Together, these components enable robust cross-domain recommendation and improve
recommendation accuracy in the target domain under strictly non-overlapping
training settings.

\section{Contributions}
The main contributions of this thesis are as follows:
\begin{itemize}
	\item We introduce a novel framework for cross-domain recommendation under non-overlapping settings by modeling
	\item We propose representing user preferences as Gaussian Mixture Models (GMMs) to capture the uncertainty and diversity of user behavior more effectively.
	\item We integrate optimal transport techniques to align user preference distributions across domains, enabling knowledge transfer
	\item We conduct extensive experiments on Amazon datasets across multiple domains to evaluate the performance of our proposed method. The results demonstrate that our approach significantly outperforms existing cross-domain recommendation techniques and traditional single-domain methods in terms of recommendation accuracy.
	\item We perform ablation studies to assess the effectiveness of representing user preferences as distributions and the impact of boosting target domain recommendation performance using source domain information.
	\item We provide insights into the benefits of modeling user preferences as distributions and leveraging optimal transport for cross-domain knowledge transfer, highlighting the potential of our approach for addressing challenges in recommender systems.
\end{itemize}

\section{Thesis Organization}
The remainder of this thesis is organized as follows:
\begin{itemize}
	\item In Chapter~\ref{chap:preliminaries}, we provide an overview of the fundamental concepts and techniques relevant to this research, including recommender systems, cross-domain recommendation, Gaussian Mixture Models, and optimal transport.
	\item In Chapter~\ref{chap:related_work}, we review related work on recommender systems, cross-domain recommendation, and optimal transport techniques.
	\item In Chapter~\ref{chap:proposed_method}, we present our proposed framework for cross-domain recommendation under non-overlapping settings, detailing the modeling of user preferences as GMMs and the integration of optimal transport for knowledge transfer.
	\item In Chapter~\ref{chap:experiments}, we describe the experimental setup, datasets, and evaluation metrics used to assess the performance of our proposed method.
	\item In Chapter~\ref{chap:results}, we present and discuss the experimental results, including comparisons with existing methods and ablation studies.
	\item Finally, in Chapter~\ref{chap:conclusion}, we summarize the main findings of this thesis and discuss potential directions for future research.
\end{itemize}
\chapter{Preliminaries}\label{chap:premilinaries}

\section{Recommender Systems}
% Mathematical modeling of the task of recommender systems
Recommender systems can be mathmatically modeled as a function $R: U \times I \rightarrow S$, where $U$ is the set of users, $I$ is the set of items, and $S$ is the set of possible scores or ratings. The goal of a recommender system is to predict the score $s \in S$ that a user $u \in U$ would give to an item $i \in I$. This prediction can be represented as $\hat{s} = R(u, i)$. The final task of recommender systems is to generate a ranked list of items for each user based on the predicted scores.

\section{Cross-Domain Recommendation}
% Mathematical modeling of the task of cross-domain recommendation
Cross-domain recommendation aims to leverage user preferences and behaviors from one or more source domains to improve recommendation performance in a target domain. Formally, let $D_s$ be the source domain with user set $U_s$ and item set $I_s$, and $D_t$ be the target domain with user set $U_t$ and item set $I_t$. The objective is to learn a recommendation function $R_t: U_t \times I_t \rightarrow S$ for the target domain by utilizing information from the source domain(s) $D_s$.
The infromation that can be transferred from the source domain to the target domain includes user-item interactions, user profiles, item attributes, and latent factors learned from the source domain. The challenge in cross-domain recommendation lies in effectively transferring knowledge while addressing issues such as domain heterogeneity, data sparsity, and cold-start problems.

\section{Gaussian Mixture Models}
% Mathematical definition of Gaussian Mixture Models
A Gaussian Mixture Model (GMM) is a probabilistic model that assumes data is generated from a mixture of several Gaussian distributions. Formally, a GMM can be represented as:
\begin{equation}
	p(x) = \sum_{k=1}^{K} \pi_k \mathcal{N}(x | \mu_k, \Sigma_k)
\end{equation}
where $K$ is the number of Gaussian components, $\pi_k$ are the mixture weights satisfying $\sum_{k=1}^{K} \pi_k = 1$, and $\mathcal{N}(x | \mu_k, \Sigma_k)$ is the Gaussian distribution with mean $\mu_k$ and covariance matrix $\Sigma_k$.

\section{Optimal Transport}
% Mathematical definition of Optimal Transport, and the variant we are using in our proposed method
Optimal transport is a mathematical framework for comparing and transforming probability distributions. Given two probability distributions $\mu$ and $\nu$ defined on spaces $X$ and $Y$, respectively, the optimal transport problem seeks to find a mapping $T: X \rightarrow Y$ that minimizes the cost of transporting mass from $\mu$ to $\nu$. The cost function $c(x, y)$ quantifies the expense of moving mass from point $x \in X$ to point $y \in Y$. The optimal transport problem can be formulated as:
\begin{equation}
	\min_{T} \int_{X} c(x, T(x)) d\mu(x)
\end{equation}
subject to the constraint that the pushforward measure $T_{\#}\mu = \nu$.
Optimal transport has been widely used in various applications, including image processing, machine learning, and economics, due to its ability to capture the geometric structure of probability distributions.
In our proposed method, we are not using optimal transport in the standard sense above, but using a related concept called Wasserstein distance to measure the distance between two probability distributions. The Wasserstein distance is defined as:
\begin{equation}
	W_p(\mu, \nu) = \left( \inf_{\gamma \in \Pi(\mu, \nu)} \int_{X \times Y} c(x, y)^p d\gamma(x, y) \right)^{1/p}
\end{equation}
where $\Pi(\mu, \nu)$ is the set of all joint distributions (couplings) with marginals $\mu$ and $\nu$, and $c(x, y)$ is the cost function. The Wasserstein distance provides a meaningful way to compare probability distributions, taking into account the underlying geometry of the data.
With the distance between two probability distributions defined, we can replace the conventional pointwise distance (e.g., Euclidean distance) between vectors with the Wasserstein distance between the corresponding distributions in our proposed method.

\chapter{Related Work} \label{chap:related_work}

\section{Recommender Systems}
% The related work of single-domain recommender systems.
This section reviews the existing literature on recommender systems, categorizing the related work into several key areas based on the techniques and approaches employed. Each subsection delves into specific methodologies and advancements within the field.

\subsection{Traditional Recommender Systems}
% Include collaborative filtering, content-based filtering, hybrid methods.
In the early stages of recommender systems, traditional techniques such as collaborative filtering, content-based filtering, and hybrid methods were predominantly used.
Collaborative filtering (CF) is one of the foundational approaches in recommender systems, operating on the principle that users with similar preferences will like similar items, or that items liked by similar users will be preferred by a given user~\cite{Bobadilla_Ortega_Hernando_Gutiérrez_2013_recommender_survey,Su_Khoshgoftaar_2009_CF_survey}.

CF can be divded into memory-based and model-based methods based on how recommendations are generated. Memory-based methods utilize user-item interaction data directly to compute similarities between users or items, while model-based methods employ machine learning algorithms to learn latent factors from the interaction data.
\subsubsection{Memory-based CF}
By user or item similarities, memory-based CF can be further categorized into user-based and item-based approaches. User-based CF recommends items to a user based on the preferences of similar users, while item-based CF suggests items similar to those the user has previously liked.

GroupLens~\cite{Resnick_1994_grouplens} is one of the earliest and most influential memory-based CF systems, which introduced user-based collaborative filtering using Pearson correlation to compute user similarities. It demonstrated the effectiveness of CF in providing personalized recommendations and laid the groundwork for subsequent research in the field.

Sarwar et al.~\cite{Sarwar_2001_item_based_CF} points out that when the number of users and items is very large, user-based CF can be computationally expensive and the sparsity of the user-item interaction matrix can lead to poor recommendation quality. Noticing that users' preferences change quickly over time but items' characteristics are relatively stable, they proposed an item-based CF approach that computes item similarities based on user interactions. The merit of this method is that item similarities can be precomputed and stored, allowing for efficient recommendation generation. And the number of items is usually much smaller than the number of users, which helps alleviate the data sparsity issue. Also, one important attribution of this paper is that it proposed a new similarity measure called adjusted cosine similarity, which accounts for individual user rating biases when computing item similarities. This method has since become a standard technique in item-based CF and has been widely adopted in various recommender systems.

\subsubsection{Model-based CF}
Model-based CF methods utilize machine learning techniques to learn latent representations of users and items from interaction data, by fitting parametric models such as matrix factorization, probabilistic latent factor models, or neural networks.
These models capture underlying preference patterns in a low-dimensional latent space, enabling generalization to unseen user–item pairs and alleviating data sparsity.

Breese et al.~\cite{Breese_Heckerman_Kadie_2013_CF_analysis} conducted a comprehensive analysis of both memory-based and model-based CF methods. This paper is the first to systematically distinguish between memory-based and model-based CF approaches, providing a detailed comparison of their strengths and weaknesses. The model-based methods mentioned in this paper include Bayesian Clustering and Bayesian Networks. Bayesian Clustering groups users into clusters based on their preferences, while Bayesian Networks model the probabilistic relationships between users and items.

Ungar and Foster~\cite{Ungar_Foster_1998_clustering_CF} pointed out that traditional clustering-based collaborative filtering methods suffer from instability and poor generalization when interaction data are highly sparse, as approaches based on KNN or simple K-means clustering rely heavily on local similarity patterns.
To address this issue, they proposed Gibbs clustering for collaborative filtering, a probabilistic co-clustering approach that jointly clusters users and items into latent classes and models their interactions through class-level link probabilities. By explicitly formulating a generative model and employing Gibbs sampling for inference, their method enforces global consistency in user and item assignments and provides a principled alternative to heuristic clustering.

Koren et al.~\cite{Koren_Bell_Volinsky_2009_MF_recommender} systematically review matrix factorization techniques for recommender systems, demonstrating that latent factor models with bias, implicit feedback, and temporal dynamics achieve consistently superior accuracy and scalability over neighborhood-based methods, and establishing matrix factorization as a dominant model-based collaborative filtering paradigm.

Salakhutdinov and Mnih introduce Probabilistic Matrix Factorization as a scalable latent factor model for large, sparse recommender systems, and further extend it to a fully Bayesian framework using MCMC, significantly improving robustness and generalization—especially for infrequent users—while establishing PMF as a foundational model-based collaborative filtering paradigm~\cite{Mnih_Salakhutdinov_2007_PMF,Salakhutdinov_Mnih_2008_Bayesian_PMF}.

While most model-based collaborative filtering methods represent user preferences as point embeddings in a latent space, several approaches instead characterize user preferences in a probabilistic manner. Since our proposed method also adopts a distributional representation of user preferences, we briefly review related works along this line.

Hofmann~\cite{Hofmann_2004_PLSA_recommender} proposed probabilistic latent semantic analysis (PLSA) for collaborative filtering, formulating user–item interactions as a latent class mixture model. In PLSA, each user is associated with a probability distribution over latent topics (or communities), and each interaction is generated by first sampling a latent topic and then drawing an item conditioned on that topic. As a result, user preferences are represented as distributions over latent semantic factors, allowing different interactions of the same user to be explained by different latent causes, rather than being tied to a single latent representation.

Marlin~\cite{Marlin_2003_user_rating_profiles} proposed the User Rating Profile (URP) model, a probabilistic latent variable approach for rating-based collaborative filtering that explicitly models uncertainty in user preferences. URP represents each user as a mixture over latent user attitudes, where the mixture proportions are drawn from a Dirichlet distribution. For each item, a latent attitude is sampled and the corresponding rating is generated according to an attitude-specific rating distribution.
By modeling users as distributions over latent preference patterns rather than fixed point representations, URP enables different items rated by the same user to be explained by different latent factors and allows direct inference of rating distributions for unseen items.

Blei, Ng, and Jordan~\cite{Blei_Ng_Jordan_2003_LDA} proposed Latent Dirichlet Allocation (LDA), a hierarchical generative probabilistic model that represents each document as a mixture over latent topics, where the topic proportions are drawn from a Dirichlet prior. By introducing a document-level latent variable, LDA provides a fully generative framework that enables principled inference for previously unseen data.
Although originally developed for text modeling, LDA has been extended to collaborative filtering by treating users as documents and items as words. Under this formulation, user preferences are modeled as probability distributions over latent topics, allowing each user–item interaction to be explained by different latent factors. This distributional representation enables LDA to capture heterogeneous user interests more flexibly than single-vector latent representations.

\subsubsection{Content-based Filtering}
Content-based filtering (CBF) recommends items to users by modeling user preferences from the attributes of items they have previously interacted with.
Typically, both users and items are represented in a shared feature space, where recommendations are generated based on the similarity between user profiles and item representations.
Since CBF relies solely on individual user history, it is less affected by user–user interaction sparsity but often suffers from limited diversity and difficulty in capturing evolving or complex user interests.

Salton et al.~\cite{Salton_Buckley_Fox_1983_auto_query_formulation} proposed the vector space model for information retrieval, in which both documents and queries are represented as weighted term vectors.
By assigning importance weights to terms (e.g., inverse document frequency) and computing similarity scores between query and document vectors, the model enables ranked retrieval based on relevance.
This representation and similarity-matching paradigm laid the foundation for content-based recommender systems, where user profiles and item content are similarly modeled in a shared feature space.

Pazzani and Billsus~\cite{Pazzani_Billsus_1997_user_profiles} proposed a content-based recommender system that learns user profiles from explicit user feedback on item content.
Their method represents items using content features and employs a naive Bayesian classifier to incrementally learn and revise user preference profiles, enabling the system to predict the interestingness of unseen items.

\subsubsection{Hybrid Methods}
Hybrid recommender systems combine both collaborative filtering and content-based filtering techniques to leverage the strengths of each approach and mitigate their respective weaknesses.
By jointly exploiting user–item interaction patterns and item content information, hybrid methods can alleviate issues such as data sparsity and cold-start that commonly affect pure collaborative filtering models.
These approaches typically integrate multiple signals at different stages of the recommendation pipeline, resulting in more robust and accurate predictions.

Burke~\cite{Burke_2002_hybrid_recommender} presented a comprehensive survey of hybrid recommender systems, systematically categorizing hybridization strategies such as weighted, switching, mixed, and feature combination approaches.
The survey analyzed how different hybrid designs integrate multiple recommendation techniques to balance their respective strengths and weaknesses, demonstrating that hybrid methods can effectively improve recommendation accuracy and alleviate issues such as data sparsity and cold-start problems.

Pazzani~\cite{Pazzani_1999_framework_CDR} proposed a unified framework for integrating collaborative, content-based, and demographic filtering methods in recommender systems.
By exploiting multiple sources of information, including user–item interactions, item content, and user profiles, the framework combines recommendations from different models to improve precision.
Experimental results demonstrated that hybrid approaches within this framework consistently outperform single-method recommenders.

Melville et al.~\cite{Melville_Mooney_Nagarajan_2002_content_boosted_CF} proposed a content-boosted collaborative filtering (CBCF) framework that integrates content-based prediction into the collaborative filtering process.
Specifically, a content-based predictor is first used to generate pseudo ratings for unrated items, producing a dense pseudo user–item matrix on which collaborative filtering is subsequently applied.
By alleviating sparsity and the first-rater problem, this approach achieves significantly improved recommendation accuracy compared to pure collaborative, pure content-based, and naive hybrid methods.

Billsus et al.~\cite{Billsus_Pazzani_2000_user_modeling} developed a hybrid news recommender system for adaptive news access that integrates collaborative filtering and content-based filtering techniques.
Their system learns personalized user models from both explicit and implicit user feedback, and combines short-term and long-term interest representations to adapt to users’ evolving information needs.
Deployed in a real-world news delivery environment, this work demonstrated the practical effectiveness of hybrid recommender systems in improving personalization quality without requiring additional user effort.

\subsection{Supervised Machine Learning in Recommender Systems}
Supervised machine learning techniques have also been widely applied to recommender systems to enhance recommendation accuracy.
By formulating recommendation as a regression or classification problem, these methods learn predictive models from labeled user–item interaction data using features such as user demographics, item attributes, and contextual information.
Algorithms including decision trees, support vector machines, and ensemble methods have been employed to capture complex relationships between features and user preferences.
While effective in leveraging rich side information, supervised learning-based approaches often rely heavily on feature engineering and struggle to generalize under sparse interaction settings.

Basilico and Hofmann~\cite{Basilico_Hofmann_2004_unifying_CF_content} proposed a unified supervised learning framework that integrates collaborative filtering and content-based filtering within a single prediction model.
Their approach formulates recommendation as a learning problem over user–item pairs by designing joint feature representations and kernel functions that enable simultaneous generalization across both user and item dimensions.
By incorporating user–item interaction data together with item and user attributes, the framework achieves improved recommendation accuracy compared to traditional collaborative or content-based methods.

Rendle~\cite{Rendle_2010_factorization_machines} introduced Factorization Machines (FMs), a supervised learning model that generalizes matrix factorization by modeling pairwise feature interactions through factorized parameters.
By representing user–item interactions, item attributes, and contextual information as sparse feature vectors, FMs can efficiently capture interactions in high-dimensional and highly sparse settings.
This unified formulation subsumes several state-of-the-art factorization models and has demonstrated superior performance over traditional collaborative and content-based approaches in various recommendation tasks.

Burges et al.~\cite{Burges_Shaked_Renshaw_Lazier_Deeds_Hamilton_Hullender_2005_LTR} proposed a learning-to-rank framework that directly optimizes the ordering of items rather than predicting absolute preference scores.
Their approach formulates ranking as a pairwise learning problem and introduces RankNet, which models ranking preferences using a probabilistic cost function optimized via gradient descent.
By focusing on ranking quality, this framework significantly improves recommendation effectiveness in scenarios where the relative order of items is more important than precise rating prediction.

Later, Burges~\cite{Burges_2010_ranknet_lambdarank_lambdamart} provided a comprehensive overview of learning-to-rank methods, including RankNet, LambdaRank, and LambdaMART.
RankNet formulates ranking as a pairwise probabilistic learning problem optimized via gradient descent, while LambdaRank introduces the concept of lambda gradients to directly optimize ranking metrics such as NDCG.
By combining LambdaRank with gradient-boosted decision trees, LambdaMART further improves ranking performance and has become a widely adopted approach in large-scale recommendation and information retrieval systems.

In industrial recommender systems, supervised learning techniques are widely adopted due to their strong predictive performance and flexibility in incorporating heterogeneous features.
Among these methods, decision tree-based models and ensemble learning techniques are particularly popular, as they provide a good balance between interpretability and the ability to capture complex feature interactions.

He et al.~\cite{He_Pan_Jin_Xu_Liu_Xu_Shi_Atallah_Herbrich_Bowers_2014_facebook_ads} developed a large-scale recommender system for Facebook Ads based on gradient boosting decision trees.
By modeling user preferences from rich user, item, and contextual features, their approach significantly improved ad targeting effectiveness and user engagement, demonstrating the practicality of supervised learning methods in real-world industrial recommendation scenarios.

Furthermore, advanced gradient boosting frameworks such as XGBoost~\cite{Chen_2016_XGBoost} and LightGBM~\cite{Ke_Meng_Finley_Wang_Chen_Ma_Ye_Liu_2017_LightGBM} have been widely applied in recommender systems to enhance both accuracy and scalability.
These methods leverage efficient tree-based boosting strategies to model high-order feature interactions, making them particularly suitable for large-scale recommendation tasks with sparse and high-dimensional feature spaces.

\subsection{Deep Learning-based Recommender Systems}
Deep learning-based recommender systems have significantly advanced the field by enabling end-to-end representation learning and modeling complex, non-linear user–item interactions.
Neural Collaborative Filtering (NCF) extends traditional matrix factorization by replacing fixed inner products with multi-layer perceptrons, allowing the model to learn more expressive interaction functions.
Beyond interaction modeling, convolutional neural networks (CNNs) have been widely used to extract informative representations from unstructured item content such as images and text, thereby enriching item features for recommendation.
Recurrent neural networks (RNNs) and their variants further incorporate temporal dynamics by modeling sequential user behaviors, enabling personalized recommendations that adapt to users’ evolving preferences.

\subsubsection{Neural Collaborative Filtering}
Neural Collaborative Filtering (NCF) is a deep learning-based recommendation framework that replaces the fixed inner product used in matrix factorization with neural networks to model user–item interactions.
By learning non-linear interaction functions through multi-layer perceptrons, NCF can capture more complex preference patterns than traditional collaborative filtering methods.
The framework unifies several neural architectures, including generalized matrix factorization (GMF), multi-layer perceptron (MLP), and their hybrid variant NeuMF, which have demonstrated superior performance on various recommendation benchmarks.

He et al.~\cite{He_Liao_Zhang_Nie_Hu_Chua_2017_NCF} proposed the Neural Collaborative Filtering (NCF) framework, which formulates collaborative filtering as a neural interaction learning problem.
Instead of relying on a fixed inner product as in traditional matrix factorization, NCF employs neural networks to learn flexible and non-linear user–item interaction functions directly from data.
In this framework, users and items are embedded into low-dimensional latent spaces and their representations are combined through neural architectures—such as generalized matrix factorization (GMF), multi-layer perceptrons (MLP), and their fusion model NeuMF—to capture complex interaction patterns beyond linear similarity measures.

\subsubsection{Sequential Recommender Systems}
Sequential recommender systems exploit the sequential patterns in users’ interaction histories to generate personalized recommendations.
By explicitly modeling the order and temporal dependencies of user–item interactions, these methods capture the dynamic evolution of user preferences over time and have been widely applied in domains such as e-commerce, music streaming, and video platforms.
Representative techniques for sequential recommendation include recurrent neural networks (RNNs) and their variants such as long short-term memory (LSTM) networks, as well as more recent Transformer-based architectures.

Hidasi et al.~\cite{Hidasi_Karatzoglou_Baltrunas_Tikk_2015_session_based_RNN} proposed a session-based recommender system that applies recurrent neural networks (RNNs) to model user behavior within individual sessions.
By representing a session as a sequence of item interactions and maintaining a recurrent hidden state, their approach captures both short-term and long-term dependencies in session data.
Furthermore, the authors introduced ranking-oriented loss functions tailored to recommendation tasks, enabling the model to significantly outperform traditional item-to-item and neighborhood-based baselines.

The mainstream sequential recommender systems now are Transformer-based models. Kang and McAuley~\cite{Kang_McAuley_2018_SASRec} proposed SASRec, a self-attentive sequential recommender system based on the Transformer encoder architecture.
By employing self-attention mechanisms, SASRec adaptively weighs historical items in a user’s interaction sequence, enabling the model to capture long-range dependencies while remaining efficient on sparse data.
Unlike recurrent models that summarize sequences through a single hidden state, SASRec directly attends to relevant past interactions, leading to improved recommendation accuracy and scalability in sequential recommendation tasks.

Sun et al.~\cite{Sun_Yuan_Wang_Shen_Li_Lu_2019_BERT4Rec} proposed BERT4Rec, a Transformer-based sequential recommender system that employs bidirectional self-attention to model user behavior sequences.
Unlike unidirectional sequential models such as RNN-based methods and SASRec, BERT4Rec leverages bidirectional contextual information by predicting masked items within a sequence using a Cloze-style training objective.
This design enables each item representation to incorporate both preceding and succeeding context, leading to more expressive sequence modeling and consistently improved recommendation performance across multiple benchmark datasets.

\subsubsection{Graph-based Recommender Systems}
Graph-based recommender systems model user–item interactions as graphs and apply graph neural networks (GNNs) to learn representations through neighborhood aggregation.
By propagating information along graph edges, these methods can effectively capture high-order connectivity and collaborative signals that are difficult to model with point-wise interaction functions.
Moreover, graph-based frameworks naturally support the integration of side information and heterogeneous relations, enabling richer modeling of user preferences and item characteristics.

Van den Berg et al.~\cite{VanDenBerg_Thomas_Kipf_Welling_2017_GCMC} proposed Graph Convolutional Matrix Completion (GCMC), which formulates collaborative filtering as a link prediction problem on a bipartite user–item interaction graph.
By employing a graph convolutional auto-encoder architecture, GCMC learns user and item representations through message passing on the interaction graph and reconstructs ratings via a bilinear decoder.
This approach effectively captures high-order collaborative signals and naturally incorporates side information, leading to improved recommendation performance on benchmark datasets.

Ying et al.~\cite{Ying_He_Chen_Eksombatchai_Hamilton_Leskovec_2018_GCN_recommender} proposed PinSage, a graph-based recommender system designed for web-scale applications.
PinSage combines graph neural networks with efficient random-walk-based neighborhood sampling to learn item representations that incorporate both graph structure and rich side information.
By addressing the scalability limitations of conventional GCNs, PinSage was successfully deployed in large-scale industrial systems such as Pinterest, demonstrating the effectiveness of graph-based recommendation models in real-world production environments.

Wang et al.~\cite{Wang_He_Wang_Feng_Chua_2019_NGCF} proposed Neural Graph Collaborative Filtering (NGCF), which explicitly integrates graph neural networks into collaborative filtering by modeling user–item interactions as a bipartite graph.
NGCF refines user and item embeddings through recursive message passing on the interaction graph, enabling the explicit modeling of high-order connectivity and collaborative signals, which leads to significant improvements in recommendation performance.

He et al.~\cite{He_Deng_Wang_Li_Zhang_Wang_2020_LightGCN} introduced LightGCN, a simplified graph convolutional network tailored for recommender systems.
Unlike prior GNN-based models that incorporate feature transformations and nonlinear activations, LightGCN argues that these components contribute little to collaborative filtering performance when only user and item IDs are available as input.
Accordingly, LightGCN retains only the neighborhood aggregation operation to propagate embeddings over the user–item interaction graph, significantly simplifying the model architecture while preserving the ability to capture high-order connectivity.
Extensive experiments demonstrate that LightGCN not only reduces computational complexity but also achieves superior recommendation accuracy compared to more complex GNN-based methods such as NGCF, making it a widely adopted and strong baseline in graph-based recommender system research.

Wu et al.~\cite{Wu_Tang_Zhu_Wang_Xie_Tan_2019_SR-GNN} proposed SR-GNN, a session-based recommender system that models user interaction sequences as graph-structured data and applies graph neural networks to learn item representations within sessions.
By constructing a directed session graph for each interaction sequence, SR-GNN is able to capture complex transition patterns among items that go beyond simple sequential dependencies.
This work represents an early attempt to integrate graph-based modeling with sequential recommendation, effectively combining the strengths of GNNs in capturing high-order relational information and sequential models in characterizing short-term user intent.

%Knowledge Graph-based Recommender Systems

\subsubsection{Generative Recommender Systems}
Generative recommender systems have emerged as an important research direction that models user preferences and item characteristics from a probabilistic perspective.
By leveraging generative models such as Variational Autoencoders (VAEs), Generative Adversarial Networks (GANs), and Normalizing Flows, these approaches aim to learn the underlying distribution of user–item interactions rather than deterministic point estimates.
Such distribution-aware modeling enables recommender systems to capture uncertainty in user preferences, generate diverse recommendations, and alleviate data sparsity issues.
These advantages are closely related to the motivation of our proposed method, which also represents user preferences as probability distributions to support more expressive and robust recommendation.

Liang et al.~\cite{Liang_Krishnan_Hoffman_Jebara_2018_VAE_CF} proposed a variational autoencoder (VAE)-based collaborative filtering framework for implicit feedback data.
By modeling user preferences as latent random variables and employing a multinomial likelihood, this approach provides a probabilistic formulation that captures uncertainty and multi-modal structures in user behavior.
Compared with deterministic autoencoder-based models, the VAE framework enables more expressive preference modeling and demonstrates strong empirical performance under sparse interaction settings.

Wang et al.~\cite{Wang_Yu_Zhang_Gong_Xu_Wang_Zhang_Zhang_2017_IRGAN} proposed IRGAN, a generative adversarial framework that unifies generative and discriminative models for information retrieval and recommendation tasks.
By formulating the learning process as a minimax game, the generative model aims to approximate the underlying relevance distribution over items, while the discriminative model learns to distinguish relevant from non-relevant user–item pairs.
Through adversarial training, IRGAN effectively improves recommendation performance, particularly under implicit feedback settings.

Chae et al.~\cite{Chae_Kang_Kim_Lee_2018_CFGAN} proposed CFGAN, a generic collaborative filtering framework based on generative adversarial networks.
Unlike prior GAN-based recommender systems that generate discrete item indices, CFGAN adopts vector-wise adversarial training, where the generator produces real-valued preference vectors and the discriminator distinguishes them from ground-truth interaction vectors.
This design effectively stabilizes adversarial learning and improves recommendation accuracy, especially under sparse implicit feedback settings.

Beyond VAEs and GANs, diffusion models have recently been introduced to recommender systems as a new class of generative models.
Wang et al.~\cite{Wang_Xu_Feng_Lin_He_Chua_2023_DiffRec} proposed DiffRec, a diffusion-based recommender system that models the user–item interaction generation process through iterative denoising.
By gradually corrupting user interaction histories and learning to recover the original interactions step by step, DiffRec provides a flexible and expressive framework for modeling complex preference distributions.
This work demonstrates the potential of diffusion models to overcome the limitations of traditional generative approaches and further improve recommendation performance under noisy and sparse interaction settings.

\section{Cross-Domain Recommender Systems}

Despite the success of single-domain recommender systems, challenges such as data sparsity and cold-start users remain difficult to address when interaction data are limited.
Cross-Domain Recommendation (CDR) tackles these issues by transferring knowledge from a source domain with abundant user–item interactions to a target domain where data are scarce.

The core objective of CDR is to leverage auxiliary information across domains to improve recommendation performance, under the assumption that user preferences or item characteristics exhibit certain transferable patterns.
By exploiting correlations between domains, CDR methods aim to enhance recommendation accuracy and robustness, especially in cold-start and sparse-data scenarios.

Based on whether domains share common entities, existing CDR approaches can be broadly categorized into \emph{overlapping} and \emph{non-overlapping} settings.
Overlapping CDR methods assume the existence of shared users or items across domains and utilize this overlap as a bridge for knowledge transfer in the training stage.
Typical techniques include joint matrix factorization, co-clustering, and graph-based models that align user preferences or item representations across domains.

In contrast, non-overlapping CDR methods address more challenging scenarios where no users or items are shared between domains during training.
These approaches generally rely on content features, latent representations, or learned mappings between domains to enable preference transfer without explicit entity overlap.
Non-overlapping CDR is particularly relevant in practical applications, but remains challenging due to the lack of direct correspondence between domains.

\subsection{Overlapping CDR}
%Still need to add more recent works.
%Add GNN based CDR methods.
The core idea of overlapping CDR is to leverage the shared users or items between domains to facilitate knowledge transfer.

Singh and Gordon~\cite{Singh_Gordon_2008_CMF} proposed Collective Matrix Factorization (CMF), a multi-relational matrix factorization framework that jointly factorizes multiple user–item interaction matrices across domains.
By sharing latent factors for overlapping users or items, CMF enables effective knowledge transfer between related domains and improves recommendation performance under data sparsity.
This work is widely regarded as a foundational approach for overlapping cross-domain recommendation.

Li et al.~\cite{Li_Yang_Xue_2009_transfer_learning_CF} proposed a transfer learning framework for collaborative filtering based on a rating-matrix generative model.
Their approach captures shared latent rating patterns across domains by learning a common cluster-level rating structure, enabling knowledge transfer even under severe data sparsity.
This framework provides a principled probabilistic formulation for overlapping cross-domain recommendation and significantly improves recommendation performance in the target domain.

Pan et al.~\cite{Pan_Xiang_Liu_Yang_2010_transfer_learning_CF} proposed a transfer learning framework for collaborative filtering that alleviates data sparsity by transferring knowledge from auxiliary domains.
Their method discovers shared latent structures of users and items in auxiliary data through matrix factorization and adapts these structures to the target domain via a principled regularization scheme.
By exploiting overlapping users or items across domains, this approach effectively captures transferable preference patterns and improves recommendation performance in sparse target domains.

%需要添加可以扩展画图的方法

\subsection{Non-overlapping CDR}
%Still need to add more recent works.
Non-overlapping cross-domain recommendation addresses scenarios where no users or items are shared between domains.
In the absence of explicit entity overlap, these methods aim to bridge domains by learning transferable representations from auxiliary information, such as content features, latent factors, or deep neural models.

Man et al.~\cite{Man_Shen_Jin_Cheng_2017_CDR_embedding_mapping} proposed EMCDR, an embedding-and-mapping framework for cross-domain recommendation.
The framework first learns latent user and item representations independently in each domain, and then learns a cross-domain mapping function to project embeddings from the source domain into the target domain.
By transferring preferences through latent space mapping, EMCDR enables effective knowledge transfer in non-overlapping and cold-start scenarios.

\section{Distributional User Preference Modeling}

Recent studies have recognized that user preferences are often uncertain and multi-modal, reflecting diverse and evolving interests.
However, most traditional recommender systems represent user preferences as deterministic point embeddings in a latent space, which implicitly assume a single dominant preference pattern.

To better capture preference variability, distributional user preference modeling has been proposed as an alternative paradigm that represents user interests as probability distributions.
By modeling preferences at the distribution level, these approaches provide a more expressive representation that can capture uncertainty, preference diversity, and complex user–item interaction patterns.

\subsection{Multi-interest User Modeling}

Multi-interest user modeling refers to a class of approaches that represent user preferences using multiple latent vectors or components in order to capture diverse and heterogeneous interests.

Early probabilistic models have implicitly adopted multi-interest representations by modeling user preferences as mixtures over latent components.
Probabilistic Latent Semantic Analysis (pLSA)~\cite{Hofmann_2004_PLSA_recommender} and related latent class models represent users through distributions over latent topics or preference patterns inferred from interaction data.
Latent Dirichlet Allocation (LDA)~\cite{Blei_Ng_Jordan_2003_LDA} further introduces a fully generative formulation by modeling user-specific mixture weights as random variables, enabling users to exhibit multiple interests to different degrees.
Similarly, Marlin~\cite{Marlin_2003_user_rating_profiles} models user rating profiles as mixtures of latent user attitudes, providing an early probabilistic formulation of multi-interest user preferences in collaborative filtering.

% DIN, MIND, ComiRec可以扩展画图
Zhou et al.~\cite{Zhou_Zhu_Song_Fan_Zhu_Ma_Yan_Jin_Li_Gai_2018_DIN} proposed the Deep Interest Network (DIN), which models users’ diverse interests through a target-aware attention mechanism.
Instead of compressing all historical behaviors into a fixed-length representation, DIN dynamically aggregates user behavior embeddings conditioned on the target item, allowing different interests to be activated for different recommendation candidates.
This adaptive representation enables DIN to capture the multi-faceted nature of user preferences and has been shown to achieve strong performance in large-scale industrial recommender systems.

Li et al.~\cite{Li_Liu_Wu_Xu_Zhao_Huang_Kang_Chen_Li_Lee_2019_MIND} proposed the Multi-Interest Network with Dynamic Routing (MIND), which explicitly represents each user with multiple interest vectors.
MIND employs a dynamic routing mechanism to cluster user behaviors into distinct interest representations, enabling the model to capture diverse and heterogeneous user preferences.
By decoupling interest extraction from item matching, MIND is suitable for large-scale retrieval scenarios and has been successfully deployed in industrial recommender systems.

Cen et al.~\cite{Cen_Zhang_Zou_Zhou_Yang_Tang_2020_ComiRec} proposed ComiRec, a controllable multi-interest recommender framework that explicitly models users with multiple interest representations.
ComiRec employs capsule networks or self-attention mechanisms to extract diverse user interests from behavior sequences, and introduces an aggregation module to balance recommendation accuracy and diversity.
By jointly considering multi-interest extraction and controllable aggregation, ComiRec extends prior multi-interest methods and has demonstrated strong effectiveness in large-scale industrial recommender systems.

\subsection{Probabilistic Preference Modeling}

Probabilistic preference modeling represents user preferences using probability distributions, providing a principled framework for characterizing uncertainty and variability in user behavior.
Unlike multi-interest user modeling methods that describe preferences with a finite set of deterministic representations, probabilistic approaches explicitly model preference uncertainty and continuous variations in latent user interests.
By treating user preferences as random variables, these methods offer a more expressive representation that is well suited for capturing diverse and evolving user behaviors.

Several early works, including the models proposed by Marlin~\cite{Marlin_2003_user_rating_profiles}, Hofmann~\cite{Hofmann_2004_PLSA_recommender}, and Blei et al.~\cite{Blei_Ng_Jordan_2003_LDA}, represent user preferences using probabilistic latent variable models, where user interests are characterized as distributions rather than deterministic embeddings.
While these methods primarily focus on capturing the multi-faceted nature of user interests, probabilistic preference modeling also emphasizes uncertainty estimation, which becomes particularly important under sparse or noisy interaction data.

Salakhutdinov and Mnih~\cite{Salakhutdinov_Mnih_2008_Bayesian_PMF} further advanced this direction by introducing Bayesian Probabilistic Matrix Factorization (BPMF), which models user and item latent factors as Gaussian distributions and performs Bayesian inference to explicitly account for uncertainty in preference estimation.
%这里还可以加

VAE-based methods provide another probabilistic framework for modeling uncertainty in user preferences. Liang et al.~\cite{Liang_Krishnan_Hoffman_Jebara_2018_VAE_CF} extended variational autoencoders to collaborative filtering by modeling user preferences as latent random variables and performing variational inference. By representing users with posterior distributions in the latent space, VAE-based recommender systems are able to capture both the variability and uncertainty inherent in user behavior, leading to more robust recommendation performance under sparse interaction data.

Beyond recommender system–specific models, some studies investigate representing entities as probability distributions in general representation learning.
Vilnis and McCallum~\cite{Vilnis_McCallum_2015_Gaussian_embeddings} proposed Gaussian embeddings, which represent words as Gaussian distributions in the embedding space, enabling the modeling of both semantic uncertainty and asymmetric relationships.
Although not originally designed for recommender systems, this distributional representation paradigm provides important methodological insights and has inspired subsequent work on modeling users and items as distributions in recommendation tasks.

\section{Optimal Transport for Representation Alignment}

Optimal Transport (OT) provides a principled framework for measuring and aligning probability distributions by computing an optimal transport plan that minimizes the cost of transforming one distribution into another.
In recent years, OT has been increasingly adopted in machine learning as a powerful tool for representation alignment, enabling comparisons and mappings between distributions in a geometrically meaningful way.
This distribution-level alignment capability has led to successful applications of OT in domain adaptation, generative modeling, and representation learning.

\subsection{OT Basics in ML}
Optimal Transport is widely used in machine learning as a distribution-aware alignment technique.  
Its ability to compare probability distributions while respecting their geometric structure has led to successful applications in domain adaptation and representation learning.

Villani~\cite{Villani_2008_optimal_transport} systematically developed the theoretical foundations of optimal transport, formalizing the Kantorovich relaxation of the Monge problem and introducing Wasserstein distances as principled metrics between probability measures. 
By framing distribution comparison as a cost-minimizing mass transportation problem, this work established a rigorous mathematical basis for measuring and aligning probability distributions, which later enabled the adoption of OT as a core tool in machine learning tasks such as distribution alignment, domain adaptation, and representation learning.

Peyré and Cuturi~\cite{Peyre_Cuturi_2019_computational_OT} provided a comprehensive survey of optimal transport from a computational and machine learning perspective, systematically bridging OT theory with practical algorithms. 
Their work reviewed a wide range of OT-based applications, including domain adaptation, generative modeling, and deep learning, and emphasized scalable solutions such as entropic regularization and Sinkhorn iterations. 
By demonstrating how OT can be efficiently integrated into modern learning pipelines, this survey established OT as a practical and versatile tool for distribution comparison and representation alignment in machine learning.

Cuturi and Marco~\cite{Cuturi_2013_Sinkhorn_distances} introduced the Sinkhorn algorithm, which incorporates an entropic regularization term into the optimal transport formulation to obtain a smooth and strictly convex objective. 
This regularization allows the resulting OT problem to be solved efficiently via iterative matrix scaling, leading to orders-of-magnitude speedups compared to classical linear programming solvers. 
As a result, Sinkhorn-based OT distances make optimal transport computationally feasible for large-scale machine learning applications and have become a cornerstone for OT-based representation learning and distribution alignment methods.

Genevay et al.~\cite{Genevay_Peyre_Cuturi_2018_Sinkhorn_generative_models} proposed Sinkhorn divergences as a differentiable and computationally tractable OT-based loss for training generative models. 
By introducing entropic regularization and leveraging automatic differentiation through Sinkhorn iterations, their approach enables stable optimization and efficient gradient computation at scale. 
This work demonstrated that OT-based losses can effectively improve the quality of generated samples, highlighting the potential of optimal transport as a powerful tool for generative modeling.

\subsection{OT in Domain Adaptation}
Optimal Transport has been extensively studied in domain adaptation as a principled framework for aligning feature distributions between source and target domains. 
By explicitly modeling the discrepancy between distributions and minimizing the cost of transporting probability mass across domains, OT-based methods enable effective knowledge transfer from the source domain to the target domain. 
As a result, these approaches can reduce domain shift and improve model performance in the target domain, especially when the two domains exhibit substantial distributional differences.

% 可以扩展画图
Courty et al.~\cite{Courty_Flamary_Tuia_Rakotomamonjy_2016_OT_domain_adaptation} proposed a seminal optimal-transport-based framework for unsupervised domain adaptation, in which feature distributions from the source and target domains are aligned by learning an optimal transportation plan. 
By minimizing the Wasserstein distance between empirical distributions, their method explicitly addresses domain shift at the distribution level and enables effective knowledge transfer from the source domain to the target domain. 
Moreover, the proposed regularized OT formulation allows the incorporation of class and structural information, leading to improved adaptation performance and establishing OT as a principled tool for representation alignment in domain adaptation.

Frogner et al.~\cite{Frogner_Zhang_Mobahi_Araya_Poggio_2015_Wasserstein_loss} introduced a deep learning framework that incorporates the Wasserstein distance as a loss function for representation learning and domain adaptation. 
By embedding OT-based losses into neural networks, their method enables end-to-end training while explicitly accounting for the geometric structure of the output space. 
Through an entropically regularized formulation, the proposed Wasserstein loss can be efficiently optimized via gradient-based methods, allowing effective alignment of feature distributions across domains and demonstrating the feasibility of OT within deep learning frameworks.

With the development of efficient OT solvers such as the Sinkhorn algorithm, OT-based domain adaptation methods have become computationally tractable for large-scale applications. These advances significantly improve the practical applicability of OT, and extensive empirical results have shown that OT-based approaches can effectively mitigate distributional shifts between source and target domains, leading to consistent performance gains across a wide range of domain adaptation tasks.
\chapter{Distributional Preference Modeling for Cross-Domain Recommendation}\label{chap:proposed_method}

\section{Background}

Recommender systems have achieved remarkable success in a wide range of real-world applications, such as e-commerce, online media platforms, and social networks. However, despite their effectiveness, many existing recommender systems continue to face fundamental challenges, including data sparsity and the cold-start problem, particularly when user--item interactions are limited or unevenly distributed. These challenges often lead to suboptimal recommendation performance, as traditional models struggle to accurately capture user preferences under such constraints.

Most traditional recommender systems model user preferences as fixed point embeddings in a latent space. While such representations are computationally efficient, they implicitly assume that a user's interests can be captured by a single deterministic vector. This assumption overlooks two important characteristics of real user behavior. First, user preferences are inherently uncertain, especially in sparse settings where limited observations are available. Second, user interests are often multi-faceted, spanning multiple latent aspects that cannot be adequately represented by a single point estimate. As a result, point-based representations may fail to capture the diversity and ambiguity of user preferences, leading to suboptimal recommendation performance.

Cross-domain recommendation (CDR) aims to alleviate cold-start and data sparsity issues by transferring knowledge from a data-rich source domain to a data-sparse target domain. A common strategy adopted by existing CDR methods is to exploit overlapping users or items between domains as explicit bridges for knowledge transfer. Through shared embeddings, adversarial alignment, or joint training objectives, these methods assume that at least part of the user or item space is shared across domains during training. However, such assumptions are often unrealistic in practical scenarios. In many real-world applications, user identities or item catalogs are domain-specific due to privacy constraints, platform isolation, or delayed synchronization, resulting in entirely non-overlapping users and items across domains during the training phase.

In the absence of overlapping entities, effective knowledge transfer across domains becomes substantially more challenging. Without explicit correspondences, aligning latent representations across domains requires modeling higher-level structural or distributional similarities rather than instance-level matches. Existing non-overlapping CDR methods typically rely on shared latent spaces or distribution matching techniques, but most of them still represent user preferences as discrete vectors, limiting their expressiveness and robustness under severe data sparsity.

Motivated by these limitations, we propose to model user preferences from a distributional perspective and perform cross-domain knowledge transfer at the distribution level. Specifically, we introduce \textbf{DUP-OT} (Distributional User Preference with Optimal Transport), a novel framework for cross-domain recommendation under strictly non-overlapping settings. The core idea of DUP-OT is to represent each user's preference as a probability distribution over latent preference components, rather than a single point embedding. This distributional representation explicitly captures both the uncertainty and the multi-aspect nature of user interests.

To enable effective cross-domain alignment between such distributional preferences, DUP-OT leverages optimal transport (OT) theory as a principled mechanism for measuring and aligning probability distributions across domains. By aligning distributional representations in a shared latent space, OT allows preference knowledge to be transferred from the source domain to the target domain without requiring any overlapping users or items during training. This design enables DUP-OT to bridge domain gaps at a structural level, making it particularly suitable for realistic cross-domain recommendation scenarios where explicit correspondences are unavailable.

In the following sections, we introduce the detailed architecture of DUP-OT, including its shared preprocessing stage for constructing a unified latent space, the distributional user preference modeling module based on Gaussian Mixture Models, and the optimal-transport-based alignment mechanism for cross-domain preference transfer.

\section{Methods}
\subsection{Overview}

The overall architecture of the proposed DUP-OT framework is illustrated in Figure~\ref{fig:dup-ot-architecture}. DUP-OT is designed for cross-domain recommendation under strictly non-overlapping settings and consists of three main stages: (1) Shared Preprocessing Stage, (2) User GMM Weights Learning Stage, and (3) Cross-Domain Rating Prediction Stage. These stages jointly enable distributional user preference modeling and cross-domain knowledge transfer via optimal transport.

\begin{figure}[h]
	\centering
	\includegraphics[width=0.8\textwidth]{figures/overall_structure}
	\caption{Architecture of the DUP-OT Framework}
	\label{fig:dup-ot-architecture}
\end{figure}

The scenario setup of DUP-OT involves two domains: a source domain $\mathcal{D}_S$ and a target domain $\mathcal{D}_T$. Each domain contains its own set of users and items, with \textbf{no overlapping users or items} between the two domains during training. The training set of source domain should happen before the valid and test set of target domain in time to avoid information leakage. The goal of DUP-OT is to leverage the abundant user--item interaction data in the source domain to enhance recommendation performance in the target domain, particularly under data sparsity conditions.

A core design principle of DUP-OT is to model user preferences as probability distributions rather than point embeddings. Specifically, we represent each user's preference as a Gaussian Mixture Model (GMM) in a shared latent space across both source and target domains. To make distribution-level alignment computationally feasible, we introduce the following assumption: within each domain, all users share a fixed set of GMM components, while only the mixture weights vary across users. The shared components capture the main latent preference aspects of the domain, whereas the user-specific weights reflect individual preferences over these aspects. This assumption is reasonable in practice, as users within the same domain often exhibit similar underlying preference structures, and it allows us to perform cross-domain alignment at the component level instead of the instance level.

The Shared Preprocessing Stage aims to construct a unified latent space for both domains. Given user--item interaction data accompanied by review texts, ratings, and timestamps, we first extract semantic features from review texts using a shared pre-trained BERT model. User and item embeddings are obtained by aggregating review-level features, where a time-decay function is applied to assign higher weights to more recent reviews during user embedding aggregation. As for item embeddings, we simply average all associated review embeddings.
This preprocessing pipeline is shared across domains to ensure consistency in representation. Since the resulting embeddings are high-dimensional, a shared autoencoder is trained on data from both domains to reduce dimensionality and produce compact embeddings in a common latent space, which serves as the basis for subsequent preference modeling.

Based on the reduced item embeddings, we then determine the fixed GMM components for each domain. Specifically, a Gaussian Mixture Model is fitted to all item embeddings within a domain, and the learned mixture components are treated as domain-level latent preference aspects. This design is motivated by the observation that items in a domain naturally reflect its major semantic and preference dimensions.

In the User GMM Weights Learning Stage, DUP-OT learns personalized preference distributions for individual users. For each domain, we train a user-specific GMM weight learner and a rating prediction model using only data from that domain. The weight learner, implemented as a multi-layer perceptron (MLP), maps a user's reduced embedding to a set of mixture weights over the fixed GMM components. These weights define the user's preference distribution. A separate rating prediction MLP then estimates user--item ratings based on weighted negative Mahalanobis distances between item embeddings and the GMM components. This stage produces expressive distributional representations of user preferences within each domain.

Finally, the Cross-Domain Rating Prediction Stage enables knowledge transfer from the source domain to the target domain via optimal transport. Since GMM components are fixed within each domain, we formulate optimal transport at the component level to align source-domain and target-domain GMMs. The resulting transport plan specifies how preference mass should be transferred between components across domains. Using this transport plan, user-specific GMM weights learned in the source domain can be mapped to the target domain, yielding adapted preference distributions. These transferred distributions are optionally fused with original target-domain user distributions to enhance preference modeling. The final user preference distributions are then used by the target-domain rating prediction model to generate improved rating predictions.

\subsection{Shared Preprocessing Stage}
The Shared Preprocessing Stage aims to extract consistent user and item embeddings from raw review data across both source and target domains. This stage consists of three main steps: (1) Review Text Embedding, (2) User and Item Embedding Aggregation, and (3) Dimensionality Reduction via Autoencoder.
As our settings involve two domains with entirely non-overlapping users and items, it is crucial to ensure that the extracted embeddings are comparable and lie in a shared latent space.
Also, the potential connections between the two domains can only be established through semantic similarities, so leveraging review texts is essential for capturing meaningful representations.

\subsubsection{Review Text Embedding}
To extract semantic features from review texts, we utilize a shared pre-trained BERT model across both domains. Specifically, we are using all-MiniMLM-L6-v2 model from Sentence-Transformers library~\cite{Reimers_Gurevych_2019_SBERT}, which is a lightweight variant of BERT optimized for generating sentence embeddings.
Given a review text, we tokenize it and feed it into the BERT model to obtain a fixed-length embedding vector that captures its semantic content. By using a shared BERT model, we ensure that review embeddings from both domains are generated in the same semantic space, facilitating cross-domain alignment later on.

\subsubsection{User and Item Embedding Aggregation}
Usually, the recent reviews of a user are more indicative of its current preferences or characteristics. To account for this temporal aspect, we apply a time-decay function when aggregating review embeddings into user embeddings.
For each user, we aggregate review-level embeddings into an initial user representation using a time-aware weighted pooling strategy.

Given a set of reviews with timestamps $\{t_i\}_{i=1}^N$ and corresponding embeddings $\{\mathbf{e}_i\}_{i=1}^N$, we define the reference time as the most recent review timestamp $t_{\text{ref}} = \max_i t_i$.
The temporal distance of each review is measured in months and truncated by a maximum value :
\[
	\Delta_i = \min\left( \frac{t_{\text{ref}} - t_i}{T}, \Delta_{\max} \right),
\]
where $T = 30 \times 86400$.
Each review is assigned an exponentially decayed weight:
\[
	w_i = \exp(-\lambda \Delta_i),
\]
which is further normalized as $\tilde{w}_i = w_i / \sum_j w_j$.
The final user embedding is computed as a weighted sum of review embeddings:
\[
	\mathbf{u} = \sum_{i=1}^N \tilde{w}_i \mathbf{e}_i .
\]

For item embedding aggregation, we assume that item attributes are relatively stable over time. Therefore, we simply compute the item representation by averaging all associated review embeddings.
Formally, the item embedding is computed as
\[
	\mathbf{v} = \frac{1}{N}\sum_{i=1}^N \mathbf{e}_i,
\]
where $\{\mathbf{e}_i\}$ are the embeddings of reviews associated with the item.

An example of item embeddings after aggregation is illustrated in Figure~\ref{fig:item_embeddings_after_aggregation}.
It can be observed that item embeddings from different domains exhibit distinct distributional patterns, indicating a substantial domain discrepancy.
Addressing this discrepancy is critical for effective cross-domain recommendation and constitutes the main focus of the following sections.

\begin{figure}[h]
	\centering
	\includegraphics[width=0.6\textwidth]{figures/domain_discrepancy.pdf}
	\caption{Item Embeddings after Aggregation}
	\label{fig:item_embeddings_after_aggregation}
\end{figure}

\subsubsection{Dimensionality Reduction via Autoencoder}
The aggregated user and item embeddings are typically high-dimensional, which can lead to increased computational costs and potential overfitting in subsequent modeling stages. To address this issue, we employ a shared autoencoder to reduce the dimensionality of embeddings from both domains.
The autoencoder consists of an encoder network that maps input embeddings to a lower-dimensional latent space and a decoder network that reconstructs the original embeddings from the latent representations. The structure of the autoencoder is illustrated in Figure~\ref{fig:autoencoder_structure}.

\begin{figure}[h]
	\centering
	\includegraphics[width=0.6\textwidth]{figures/autoencoder_structure.pdf}
	\caption{Autoencoder Structure}
	\label{fig:autoencoder_structure}
\end{figure}

The autoencoder is trained on a combined dataset of user and item embeddings from both source and target domains.
The training objective is to minimize the reconstruction loss, defined as the mean squared error between the original embeddings and their reconstructions.
By sharing the autoencoder across domains, we ensure that the resulting reduced embeddings lie in a common latent space, which is essential for following distributional preference modeling and cross-domain alignment.

After training, we obtain reduced user and item embeddings by passing the original embeddings through the encoder network. These reduced embeddings serve as the basis for subsequent GMM-based preference modeling and optimal transport alignment in the DUP-OT framework.

\subsection{User GMM Weights Learning Stage}
In the User GMM Weights Learning Stage, we aim to learn expressive distributional representations of user preferences in the shared latent space constructed by the Shared Preprocessing Stage.
Specifically, each user's preference is modeled as a Gaussian Mixture Model (GMM), which enables the representation of multi-aspect user interests in a probabilistic manner.
This stage consists of two key components: (1) determining a fixed set of domain-level GMM components, and (2) learning user-specific mixture weights together with a corresponding rating prediction model.

\subsubsection{Fixed GMM Component Determination}
To capture the major latent preference aspects within each domain, we first construct a domain-level Gaussian Mixture Model (GMM) by fitting it to the reduced item embeddings obtained from the Shared Preprocessing Stage.
The resulting GMM components are treated as fixed and shared by all users within the same domain, serving as latent preference components for subsequent user-specific preference modeling.

To build intuition for Gaussian Mixture Models, we first present a one-dimensional (1D) example in Figure~\ref{fig:gmm_1d}, where the roles of individual Gaussian components and mixture weights can be easily visualized.
We then illustrate a two-dimensional (2D) GMM in Figure~\ref{fig:gmm_2d} to demonstrate how mixture models generalize from scalar to vector-valued representations.

\begin{figure}[h]
	\centering
	\includegraphics[width=0.6\textwidth]{figures/GMM_1D.pdf}
	\caption{1D Gaussian Mixture Model Example}
	\label{fig:gmm_1d}
\end{figure}

\begin{figure}[h]
	\centering
	\includegraphics[width=0.6\textwidth]{figures/GMM_2D.pdf}
	\caption{2D Gaussian Mixture Model Example}
	\label{fig:gmm_2d}
\end{figure}

To this end, we adopt the \texttt{BayesianGaussianMixture} implementation from the scikit-learn library~\cite{Pedregosa_2011_scikit-learn} to fit GMMs on item embeddings.
Unlike conventional GMMs that require manually specifying the number of mixture components, this approach is based on variational Bayesian inference with a Dirichlet prior over mixture weights.
Such a formulation allows mixture components that are not sufficiently supported by the data to be automatically suppressed during training, enabling the effective number of active components to be inferred directly from the data.

Formally, given a set of item embeddings $\{\mathbf{x}_n\}_{n=1}^N$, the Bayesian Gaussian Mixture Model assumes the following generative process:
\[
	\boldsymbol{\pi} \sim \mathrm{Dirichlet}(\boldsymbol{\alpha}), \qquad
	z_n \sim \mathrm{Categorical}(\boldsymbol{\pi}), \qquad
	\mathbf{x}_n \mid z_n = k \sim \mathcal{N}(\boldsymbol{\mu}_k, \boldsymbol{\Sigma}_k),
\]
where $\boldsymbol{\pi}$ denotes the mixture weights, $z_n$ is the latent component assignment, and $(\boldsymbol{\mu}_k, \boldsymbol{\Sigma}_k)$ represent the mean and covariance of the $k$-th Gaussian component.

Exact posterior inference in this model is intractable. Therefore, variational Bayesian inference is employed by maximizing the evidence lower bound (ELBO):
\[
	\mathcal{L}(q) =
	\mathbb{E}_{q}\!\left[\log p(\mathbf{X}, \mathbf{Z}, \boldsymbol{\pi}, \boldsymbol{\mu}, \boldsymbol{\Sigma})\right]
	-
	\mathbb{E}_{q}\!\left[\log q(\mathbf{Z}, \boldsymbol{\pi}, \boldsymbol{\mu}, \boldsymbol{\Sigma})\right],
\]
where $q(\cdot)$ denotes the variational posterior distribution.

Due to the Dirichlet prior imposed on the mixture weights, components that are weakly supported by the data are assigned negligible posterior mass, effectively pruning redundant components.
As a result, the Bayesian GMM provides a principled and flexible mechanism for determining the effective number of domain-level latent preference components.

\subsubsection{User-Specific GMM Weight Learning and Rating Prediction}

With the fixed GMM components determined for each domain, we proceed to learn user-specific mixture weights that characterize individual preference distributions in the shared latent space.
For each domain, DUP-OT employs two neural modules trained solely on the data from that domain: a user-specific GMM weight learner and a rating prediction network.
These two modules are optimized jointly to ensure that the learned preference distributions are directly aligned with the rating prediction objective.

\paragraph{User-specific GMM weight learning.}
Let $\mathbf{u}_i \in \mathbb{R}^d$ denote the reduced embedding of user $i$ obtained from the Shared Preprocessing Stage.
Given the fixed set of domain-level GMM components $\{(\boldsymbol{\mu}_k,\boldsymbol{\Sigma}_k)\}_{k=1}^{K}$, we learn personalized mixture weights via a multi-layer perceptron (MLP):
\[
	\mathbf{s}_i = f_{\theta}(\mathbf{u}_i) \in \mathbb{R}^{K}, \qquad
	\boldsymbol{\pi}_i = \mathrm{softmax}(\mathbf{s}_i),
\]
where $\boldsymbol{\pi}_i = [\pi_{i1}, \ldots, \pi_{iK}]^\top$ represents the user-specific mixture weights satisfying $\sum_{k=1}^{K} \pi_{ik} = 1$ and $\pi_{ik} \ge 0$.
The resulting weights define a personalized preference distribution over the domain-level latent preference components.

\paragraph{Distribution-aware user--item representation.}
For an item $j$ with reduced embedding $\mathbf{v}_j \in \mathbb{R}^d$, we compute its Mahalanobis distance to each GMM component:
\[
	d_{ijk} = \sqrt{(\mathbf{v}_j - \boldsymbol{\mu}_k)^\top \boldsymbol{\Sigma}_k^{-1} (\mathbf{v}_j - \boldsymbol{\mu}_k)}.
\]
These component-wise distances are first negated and then combined with the user-specific mixture weights to construct a distribution-aware representation of user--item compatibility:
\[
	\mathbf{h}_{ij} = \left[ \pi_{i1} (-d_{ij1}), \; \pi_{i2} (-d_{ij2}), \; \ldots, \; \pi_{iK} (-d_{ijK}) \right]^\top \in \mathbb{R}^{K}.
\]
This formulation enables user preferences and item representations to be compared in a distribution-aware manner by jointly considering latent preference components and their personalized importance.

\paragraph{Rating prediction and joint training.}
A separate MLP is employed to predict the rating for a given user--item pair:
\[
	\hat{r}_{ij} = g_{\phi}(\mathbf{h}_{ij}),
\]
where $g_{\phi}(\cdot)$ denotes the rating prediction network.
The GMM weight learner $f_{\theta}$ and the rating prediction network $g_{\phi}$ are trained jointly in a supervised fashion by minimizing the rating prediction loss over the observed interactions $\mathcal{D}$:
\[
	\mathcal{L}_{\mathrm{rate}}(\theta, \phi) =
	\frac{1}{|\mathcal{D}|}
	\sum_{(i,j,r_{ij}) \in \mathcal{D}}
	\left( \hat{r}_{ij} - r_{ij} \right)^2,
\]
where $r_{ij}$ and $\hat{r}_{ij}$ denote the ground-truth and predicted ratings, respectively.
Through joint optimization, the learned user-specific preference distributions are directly tailored to the recommendation task.

\subsection{Cross-Domain Rating Prediction Stage}

In the Cross-Domain Rating Prediction Stage, we leverage optimal transport (OT)
theory to facilitate knowledge transfer from the source domain to the target
domain at the distribution level.
It is important to emphasize that the non-overlapping constraint in our setting
is imposed strictly during the training stage: no overlapping users or items are
exploited when learning user preference models or cross-domain alignments.
During inference and evaluation, however, users may appear in both domains, which
reflects realistic deployment scenarios and allows transferred knowledge to be
utilized in a principled manner.
This design is reasonable because, in many real-world applications, cross-domain
user correspondences are unavailable or restricted during training due to privacy
concerns, system isolation, or data access limitations, while such correspondences
may become observable at deployment or evaluation time.

Based on the User GMM Weights Learning Stage, each user's preference in both the
source and target domains is represented as a Gaussian Mixture Model (GMM) defined
over fixed, domain-level latent preference components.
Given these distributional representations, we formulate cross-domain knowledge
transfer as an optimal transport problem between the source-domain and
target-domain GMM component sets.
Since all users within a domain share the same set of GMM components, OT is
performed at the component level rather than the instance level, enabling
efficient and stable alignment of latent preference aspects across domains.

\paragraph{Optimal transport formulation and cost matrix.}
Formally, let $\boldsymbol{\pi}^{s}\in\mathbb{R}_+^{K_s}$ and
$\boldsymbol{\pi}^{t}\in\mathbb{R}_+^{K_t}$ denote the mixture weight vectors over
the fixed GMM components in the source and target domains, respectively.
We define a cost matrix $\mathbf{C}\in\mathbb{R}^{K_s\times K_t}$, where each entry
$C_{kl}$ measures the dissimilarity between the $k$-th source-domain component and
the $l$-th target-domain component.

Specifically, the cost matrix is constructed using the 2-Wasserstein distance
between Gaussian distributions.
For the $k$-th source-domain component
$\mathcal{N}(\boldsymbol{\mu}_k^{s}, \boldsymbol{\Sigma}_k^{s})$
and the $l$-th target-domain component
$\mathcal{N}(\boldsymbol{\mu}_l^{t}, \boldsymbol{\Sigma}_l^{t})$, the cost is defined as:
\[
	C_{kl} =
	\lVert \boldsymbol{\mu}_k^{s} - \boldsymbol{\mu}_l^{t} \rVert_2^2
	+
	\operatorname{Tr}\!\left(
	\boldsymbol{\Sigma}_k^{s} + \boldsymbol{\Sigma}_l^{t}
	- 2\left(
	(\boldsymbol{\Sigma}_k^{s})^{1/2}
	\boldsymbol{\Sigma}_l^{t}
	(\boldsymbol{\Sigma}_k^{s})^{1/2}
	\right)^{1/2}
	\right).
\]

Figure~\ref{fig:cost_matrix_illustration} provides an intuitive illustration of
how the cost matrix is constructed by computing pairwise Wasserstein distances
between GMM components in the source and target domains.
Each entry in the cost matrix corresponds to the transport cost between a pair of
latent preference components, serving as the fundamental unit for subsequent OT
alignment.

\begin{figure}[h]
	\centering
	\includegraphics[width=0.6\textwidth]{figures/cost_matrix_calculation.pdf}
	\caption{Cost Matrix Construction via Pairwise Wasserstein Distances}
	\label{fig:cost_matrix_illustration}
\end{figure}

Given the cost matrix, the optimal transport plan $\mathbf{T}$ is obtained by
solving:
\[
	\min_{\mathbf{T}\ge 0}\ \langle \mathbf{T}, \mathbf{C}\rangle
	\quad \text{s.t.}\quad
	\mathbf{T}\mathbf{1}=\boldsymbol{\pi}^{s},\
	\mathbf{T}^\top\mathbf{1}=\boldsymbol{\pi}^{t}.
\]

When no regularization is applied, this formulation corresponds to the Earth
Mover's Distance (EMD), which can be solved as a linear programming problem to
obtain an exact and typically sparse transport plan.
Alternatively, we adopt the entropically regularized OT formulation:
\[
	\min_{\mathbf{T}\ge 0}\ \langle \mathbf{T}, \mathbf{C}\rangle
	+ \varepsilon \sum_{k,l} T_{kl}(\log T_{kl}-1),
\]
which can be efficiently solved using the Sinkhorn algorithm.
The entropy regularization yields a smooth and dense transport plan while
significantly reducing computational cost.
In our setting, OT is performed at the component level, where the number of GMM
components is relatively small.
Therefore, both EMD and Sinkhorn-based OT are computationally feasible.
In practice, we primarily adopt the Sinkhorn algorithm for efficiency and
numerical stability, while EMD is used for analysis when an exact transport plan is
desired.

\paragraph{Distribution transfer and fusion.}
Using the learned optimal transport plan, we transfer user-specific mixture
weights from the source domain to the target domain as:
\[
	\tilde{\boldsymbol{\pi}}_i^{t} = \mathbf{T}^\top \boldsymbol{\pi}_i^{s}.
\]

At inference time, we construct the final target-domain preference distribution
via a linear fusion strategy:
\[
	\boldsymbol{\pi}_i^{\mathrm{final}} =
	\lambda_i \boldsymbol{\pi}_i^{t} + (1-\lambda_i)\tilde{\boldsymbol{\pi}}_i^{t},
\]
where $\lambda_i \in [0,1]$ controls the contribution of the original
target-domain distribution.
If a user has interactions in both domains, $\lambda_i \in (0,1)$ is used to
combine domain-specific evidence with transferred knowledge.
If a user has interactions only in the source domain, $\lambda_i = 0$ and the
final distribution relies solely on the transferred preference.
If a user has interactions only in the target domain, $\lambda_i = 1$ and the
original target-domain distribution is retained.

Finally, the fused user preference distributions are fed into the target-domain
rating prediction model to generate enhanced rating predictions.
Through this training-time non-overlapping and inference-time adaptive fusion
design, DUP-OT faithfully reflects practical cross-domain recommendation
scenarios while maintaining a strict non-overlapping assumption during model
learning.
\chapter{Experiments} \label{chap:experiments}

In this chapter, we present the detailed setup of our experiments to evaluate the
proposed \textbf{DUP-OT} framework for non-overlapping cross-domain recommendation training.
Specifically, we first proposed the research questions we want to solve by conducting experiments,
and then we'll describe the datasets we used and how we preprocessed them, the
experiments we conducted, including the baseline models we compared against, the evaluation metrics we employed.
Finally, we outline the implementation details of our proposed method, including
hyperparameter settings and training procedures to ensure reproducibility.

The final results and analyses will be presented in the next chapter.

\section{Research Questions}
As the proposed \textbf{DUP-OT} framework introduces two key components: (1) distribution-based user preference modeling using GMMs, and (2) cross-domain preference transfer via optimal transport,
we want to design experiments to figure out the effects of these two components respectively.
Moreover, we also want to compare our proposed method with existing non-overlapping cross-domain recommendation methods.
Thus, the research questions we aim to answer through our experiments can be summarized as follows:
\begin{itemize}
	\item \textbf{RQ1:} Does introducing cross-domain information enhance the recommendation performance in the target domain?
	\item \textbf{RQ2:} Does modeling user preferences as distributions, rather than vectors, lead to improved recommendation performance?
	\item \textbf{RQ3:} How does the performance of our proposed method compare with that of existing non-overlapping CDR models?
\end{itemize}

To answer \textbf{RQ1}, we conduct an ablation study comparing target-domain recommendation performance with and without leveraging
source-domain information. When we do not use source-domain data, our model essentially reduces to a single-domain recommendation model that employs GMMs for user preference modeling.

To address \textbf{RQ2}, we evaluate our target-domain model without source-domain information against
some single-domain baseline models that represent user preferences as vectors.
The idea is, if we don't use source-domain data, the target-domain model will reduce to a single-domain recommendation model,
and thus the only difference between our model and the baselines will be the way of user preference modeling.
Thus we can isolate the effect of distribution-based user preference modeling.

Finally, to answer \textbf{RQ3}, we simply compare our full model with representative non-overlapping CDR baseline models
in the same experimental settings.

\section{Datasets and Preprocessing}
We set our experiment under an e-commerce scenario, where different product categories are treated as different domains.
For example, Electronics and Video Games are two different domains, and our target is to leverage user interactions in the source domain (e.g., Video Games)
to improve recommendation performance in the target domain (e.g., Electronics).
To this end, we choose the Amazon Review dataset~\cite{Ni_Li_McAuley_2019_Justifying_Recommendations} for our experiments,
specifically, we choose the 5-core version of four domains: Electronics, Digital Music, Movies \& TV, and Video Games.
To make sure the 5-core really holds in these four datasets, we further filter the datasets to ensure that each user and item has at least 5 interactions.
After preprocessing, the statistics of the four datasets are summarized in Table~\ref{tab:dataset_stats}.

\begin{table}[h]
	\centering
	\caption{Statistics of the Amazon Review datasets after preprocessing.}
	\label{tab:dataset_stats}
	\begin{tabular}{lrrrr}
		\toprule
		Domain        & Users   & Items   & Interactions & Density (\%) \\
		\midrule
		Electronics   & 728,489 & 159,729 & 6,737,580    & 0.0058       \\
		Digital Music & 16,252  & 11,269  & 166,942      & 0.0912       \\
		Movies \& TV  & 297,377 & 59,925  & 3,408,612    & 0.0191       \\
		Video Games   & 55,144  & 17,286  & 496,904      & 0.0521       \\
		\bottomrule
	\end{tabular}
\end{table}

Considering the interaction density and the relative time stamps of the four domains,
we select Electronics as the target domain, and the other three domains as source domains in our experiments.

Although our proposed method is not a sequential recommendation model, we think it's still important to avoid data leakage
when splitting the datasets into training, validation, and test sets. So we sort the interactions of each user by their time stamps,
and then split target-domain interactions, in our case, Electronics, into training, validation, and test sets globally by the ratio of 80\%, 10\%, and 10\%.
Then to avoid data leakage, we need to ensure the interactions in the source domain used for training happened before
the earliest interaction in the target-domain validation set. To this end, we first find the earliest time stamp $t_{val}$ in the target-domain validation set,
and then filter the source-domain interactions to only keep those happened before $t_{val}$.
To reduce computational costs, we search for $t_{val}$ in a step of 5\% in the time-stamp sorted source-domain interactions.
We don't need to split the source-domain interactions into training, validation, and test sets,
as we only use source-domain data for training in our experiments.

As we split the interactions of target-domain globally, it's possible that some users only appear in the validation or test sets.
When we build the Dataset object for training, we just randomly set the embeddings of the users who only appear in the validation or test sets as random vectors.

\section{Experimental Setup}
To answer the research questions proposed above, we mainly design three groups of experiments
corresponding to \textbf{RQ1}, \textbf{RQ2}, and \textbf{RQ3} respectively.
All experiments are conducted on the preprocessed Amazon Review datasets described in the previous section, and
the target domain is always Electronics while the source domain varies among Digital Music, Movies \& TV, and Video Games.
In other words, we conduct experiments on three source-target domain pairs:
\begin{itemize}
	\item Digital Music $\rightarrow$ Electronics
	\item Movies \& TV $\rightarrow$ Electronics
	\item Video Games $\rightarrow$ Electronics
\end{itemize}

To answer \textbf{RQ1}, which investigates the effect of cross-domain information,
we compare two variants of our proposed framework: DUP-OT (w/ source) and DUP-OT (w/o source),
which means we either leverage source-domain information for target-domain recommendation or not.
These two variants share identical architectures, representation learning strategies, and training protocols.
The only difference lies in whether source-domain preference distributions are transferred to the target domain via optimal transport
in the last stage of our framework.
Therefore, performance differences can be directly attributed to the effect of cross-domain information.

To address \textbf{RQ2}, which examines the effect of distribution-based user preference modeling,
we compare DUP-OT (w/o source) with representative single-domain recommendation models,
namely LightGCN~\cite{He_Deng_Wang_Li_Zhang_Wang_2020_LightGCN} and NeuMF~\cite{He_Liao_Zhang_Nie_Hu_Chua_2017_NCF}.
In this setting, DUP-OT (w/o source) does not leverage any source-domain interactions and is trained exclusively on the target-domain training set.
Despite this, the only difference between DUP-OT (w/o source) and the baselines is the way of user preference modeling.
Our model represents user preferences as GMMs, while the baselines use point embeddings.
Thus, performance differences can be directly attributed to the effect of distribution-based user preference modeling.

Finally, to answer \textbf{RQ3}, which compares our proposed method with existing non-overlapping CDR models,
we select a representative non-overlapping CDR baseline model, TDAR~\cite{Yu_Lin_Ge_Ou_Qin_2020_TDAR},
and compare it with our full model, DUP-OT (w/ source), in the same experimental settings.

\subsection{Baseline Models}

As discussed in the previous section, we select different baseline models
to answer the proposed research questions from complementary perspectives.
In particular, the selected baselines cover both single-domain
recommendation models and representative non-overlapping cross-domain
recommendation approaches.
In total, we consider three baseline models and two variants of our
proposed framework in the experiments, which are described as follows:
\begin{itemize}

	\item \textbf{LightGCN}~\cite{He_Deng_Wang_Li_Zhang_Wang_2020_LightGCN}:
	      LightGCN is a representative graph-based collaborative filtering model
	      that learns user and item representations by propagating embeddings on
	      the user--item interaction graph.
	      Unlike earlier GCN-based recommenders, LightGCN removes feature
	      transformation and non-linear activation, and retains only neighborhood
	      aggregation, resulting in a simple yet effective architecture.
	      As a strong single-domain baseline, LightGCN represents user preferences
	      as deterministic embedding vectors and relies solely on target-domain
	      interaction data.
	      We include LightGCN to evaluate whether distribution-based user
	      preference modeling provides advantages over state-of-the-art
	      vector-based graph collaborative filtering methods.

	\item \textbf{NeuMF}~\cite{He_Liao_Zhang_Nie_Hu_Chua_2017_NCF}:
	      NeuMF is a classical neural collaborative filtering model that unifies
	      matrix factorization and multi-layer perceptrons under a neural
	      framework.
	      By replacing the inner product with a non-linear interaction function,
	      NeuMF aims to capture complex user--item interaction patterns in the
	      latent space.
	      Similar to LightGCN, NeuMF represents users and items as point-valued
	      latent vectors and operates in a purely single-domain setting.
	      This baseline is adopted to compare our distribution-based user
	      preference modeling with a widely used neural vector-based
	      recommendation paradigm.

	\item \textbf{TDAR}~\cite{Yu_Lin_Ge_Ou_Qin_2020_TDAR}:
	      TDAR is a representative non-overlapping cross-domain recommendation
	      model based on text-enhanced domain adaptation.
	      In the absence of shared users or items across domains, TDAR leverages
	      domain-invariant textual features extracted from user reviews as anchor
	      points to align latent spaces via adversarial training.
	      By transferring distribution patterns learned from a dense source
	      domain to a sparse target domain, TDAR alleviates data sparsity in
	      non-overlapping scenarios.
	      We include TDAR as a strong non-overlapping CDR baseline to compare
	      our optimal transport-based preference transfer mechanism with
	      existing text-driven domain adaptation approaches.

	\item \textbf{DUP-OT (w/ source)}:
	      This variant corresponds to the full version of our proposed
	      \textbf{DUP-OT} framework.
	      It models user preferences as Gaussian mixture distributions and
	      transfers such distributional preferences from the source domain to
	      the target domain via optimal transport.
	      This setting evaluates the complete effectiveness of distribution-based
	      preference modeling combined with cross-domain knowledge transfer.

	\item \textbf{DUP-OT (w/o source)}:
	      This variant disables the cross-domain transfer component and trains
	      the model using only target-domain data.
	      Under this setting, DUP-OT degenerates into a single-domain
	      recommendation model that still represents user preferences as
	      probability distributions.
	      By comparing this variant with vector-based single-domain baselines,
	      we can isolate and analyze the impact of distribution-based user
	      preference modeling independent of cross-domain information.

\end{itemize}

\subsection{Evaluation Metrics}

To quantitatively assess the recommendation performance of different models,
we formulate the recommendation task as a \emph{rating prediction} problem
and employ two widely used evaluation metrics: Root Mean Square Error (RMSE)
and Mean Absolute Error (MAE).

We adopt rating prediction as the evaluation task for two main reasons.
First, rating prediction provides a fine-grained and direct measure of how
accurately a model captures user preferences, which is particularly important
in cross-domain recommendation scenarios where the goal is to transfer
preference knowledge from a source domain to a target domain.
Second, in non-overlapping cross-domain recommendation settings, ranking-based
evaluation protocols often rely on strong assumptions about negative sampling
and candidate item sets, which may introduce additional bias.
In contrast, rating prediction allows us to evaluate model performance on
observed user--item interactions without requiring assumptions about
unobserved entries.

Following prior cross-domain recommendation studies that focus on rating
prediction under non-overlapping settings, we adopt RMSE and MAE as our
primary evaluation metrics.
Both metrics have been widely used in existing cross-domain recommendation
work and provide complementary perspectives on prediction accuracy.
RMSE penalizes large prediction errors more heavily and is therefore sensitive
to outliers, while MAE measures the average absolute deviation and offers a
more robust assessment of overall prediction quality.

Formally, given a set of test user--item interactions
$\mathcal{D}_{test} = \{(u, i, r_{ui})\}$,
where $r_{ui}$ denotes the ground-truth rating of user $u$ on item $i$,
the RMSE and MAE are defined as follows:
\begin{equation}
	\text{RMSE} = \sqrt{\frac{1}{|\mathcal{D}_{test}|}
		\sum_{(u, i, r_{ui}) \in \mathcal{D}_{test}} (\hat{r}_{ui} - r_{ui})^2}
\end{equation}
\begin{equation}
	\text{MAE} = \frac{1}{|\mathcal{D}_{test}|}
	\sum_{(u, i, r_{ui}) \in \mathcal{D}_{test}} |\hat{r}_{ui} - r_{ui}|
\end{equation}
where $\hat{r}_{ui}$ denotes the predicted rating generated by a recommendation
model.
Lower values of RMSE and MAE indicate better recommendation accuracy.

\section{Implementation Details}
In this section, we provide implementation details of our proposed \textbf{DUP-OT} framework
to ensure reproducibility of our experimental results.

We implement our model using PyTorch and conduct experiments on a machine with an NVIDIA RTX 3080 GPU.
To retrieve user and item embeddings from the interaction data, we utilize a pre-trained BERT model to encode from review texts.
Specifically, we use the \texttt{all-MiniLM-L6-v2} variant from the Sentence Transformers library~\cite{Reimers_Gurevych_2019_SBERT},
which produces 384-dimensional embeddings.

To
\chapter{Results} \label{chap:results}

In this chapter, we present and analyze the experimental results of the proposed
\textbf{DUP-OT} framework. The goal of this chapter is to quantitatively and
qualitatively evaluate the effectiveness of distribution-based user preference
modeling and cross-domain preference transfer via optimal transport.

We first provide an overall comparison between DUP-OT and baseline models across
different source--target domain pairs. We then conduct detailed analyses to
answer the three research questions raised in the previous chapter, focusing on
the effects of cross-domain information, distribution-based modeling, and
comparisons with existing cross-domain recommendation methods.

\section{Overall Performance Comparison}

Table~\ref{tab:results} summarizes the rating prediction performance of all
baseline models and the proposed DUP-OT framework. The results are reported in
terms of RMSE and MAE on the target-domain test set, averaged over five random
seeds. Lower values indicate better recommendation performance.

\begin{table*}[t]
	\centering
	\caption{Rating prediction performance (RMSE and MAE) of baseline models and the proposed DUP-OT framework on different source--target domain pairs.}
	\label{tab:results}

	\begin{subtable}{\textwidth}
		\centering
		\caption{Cross-domain recommendation results with Electronics as the target domain}
		\begin{tabular}{l cc cc cc}
			\toprule
			Source $\rightarrow$ Target
			      & \multicolumn{2}{c}{Digital Music $\rightarrow$ Electronics}
			      & \multicolumn{2}{c}{Movies \& TV $\rightarrow$ Electronics}
			      & \multicolumn{2}{c}{Video Games $\rightarrow$ Electronics}                                               \\
			\cmidrule(lr){2-3}
			\cmidrule(lr){4-5}
			\cmidrule(lr){6-7}
			Model & RMSE                                                        & MAE             & RMSE & MAE & RMSE & MAE \\
			\midrule
			TDAR
			      & 1.5688                                                      & 1.0904
			      & 1.5677                                                      & 1.0896
			      & 1.5688                                                      & 1.0902                                    \\

			DUP-OT (w/o source)
			      & 1.3699                                                      & 0.9463
			      & 1.3494                                                      & 0.8806
			      & 1.3716                                                      & \textbf{0.9481}                           \\

			DUP-OT (w/ source)
			      & \textbf{1.2919}                                             & \textbf{0.8965}
			      & \textbf{1.2907}                                             & \textbf{0.8774}
			      & \textbf{1.3376}                                             & 1.0032                                    \\
			\bottomrule
		\end{tabular}
	\end{subtable}

	\vspace{1em}

	\begin{subtable}{\textwidth}
		\centering
		\caption{Single-domain recommendation performance on the Electronics dataset}
		\begin{tabular}{l cc}
			\toprule
			Model    & RMSE            & MAE             \\
			\midrule
			LightGCN & 1.5317          & 1.1179          \\
			NeuMF    & 1.4599          & 1.3297          \\
			\midrule
			DUP-OT (w/o source, average)
			         & \textbf{1.3636} & \textbf{0.9250} \\
			\bottomrule
		\end{tabular}
	\end{subtable}

\end{table*}

From Table~\ref{tab:results}, we observe that DUP-OT consistently outperforms both
single-domain and cross-domain baseline models across all evaluated settings.
In particular, DUP-OT with source-domain information achieves the lowest RMSE and
MAE in all source--target configurations, indicating the effectiveness of
cross-domain preference transfer.

\section{Effect of Cross-Domain Information}

We first analyze the impact of incorporating source-domain information on
target-domain recommendation performance, corresponding to \textbf{RQ1}. To this
end, we compare two variants of the proposed framework: DUP-OT (w/ source) and
DUP-OT (w/o source).

These two variants share identical architectures, representation learning
strategies, and training protocols. The only difference lies in whether
source-domain preference distributions are transferred to the target domain via
optimal transport. Therefore, performance differences can be directly attributed
to the effect of cross-domain information.

As shown in Table~\ref{tab:results}, DUP-OT (w/ source) consistently achieves
lower RMSE and MAE than DUP-OT (w/o source) across all source--target pairs.
For example, when transferring from Digital Music to Electronics, incorporating
source-domain information reduces RMSE from 1.3699 to 1.2919 and MAE from 0.9463
to 0.8965. Similar trends can be observed for the Movies \& TV and Video Games
source domains.

These results demonstrate that source-domain user preference distributions
provide complementary and useful signals for target-domain recommendation, even
when users and items do not overlap across domains. This finding supports the
central motivation of DUP-OT and confirms the effectiveness of optimal transport
based preference transfer.

\section{Effect of Distribution-Based User Preference Modeling}

Next, we evaluate the benefit of modeling user preferences as probability
distributions rather than point embeddings, corresponding to \textbf{RQ2}. To
isolate this effect, we compare DUP-OT (w/o source) with representative
single-domain recommendation models, namely LightGCN and NeuMF, in the
Electronics-only setting.

Notably, DUP-OT (w/o source) does not leverage any source-domain interactions and
is trained exclusively on the target-domain training set. Despite this
restriction, DUP-OT (w/o source) achieves substantially better performance than
both LightGCN and NeuMF, as shown in the lower part of Table~\ref{tab:results}.

This performance gap suggests that distribution-based user modeling enables the
model to capture uncertainty and multi-interest behavior that cannot be
effectively represented by single-vector embeddings. By representing user
preferences as mixtures over latent semantic components, DUP-OT provides a more
expressive and flexible representation, leading to improved rating prediction
accuracy.

\section{Comparison with Cross-Domain Recommendation Baselines}

We further compare DUP-OT with TDAR, a representative cross-domain recommendation
method designed for non-overlapping scenarios, corresponding to \textbf{RQ3}.
TDAR employs adversarial learning to align feature representations across domains
and has been shown to be effective in prior work.

As shown in Table~\ref{tab:results}, DUP-OT (w/ source) consistently outperforms
TDAR across all evaluated domain pairs, achieving significantly lower RMSE and
MAE. This result indicates that aligning user preference distributions via
optimal transport is more effective than adversarial feature alignment for the
task of cross-domain recommendation.

One possible explanation is that adversarial alignment primarily focuses on
matching marginal feature distributions, which may be insufficient for capturing
fine-grained preference structures. In contrast, DUP-OT aligns semantically
grounded preference distributions, enabling more precise and interpretable
cross-domain knowledge transfer.

\section{Additional Observations and Discussion}

Beyond the overall performance improvements, we observe that the benefits of
DUP-OT are particularly pronounced for users with sparse interaction histories.
This suggests that distribution-based modeling and cross-domain transfer are
especially effective in alleviating data sparsity and cold-start issues.

We also observe that the magnitude of performance improvement varies across
source domains. In general, source domains that are semantically closer to the
target domain, such as Digital Music and Movies \& TV, yield larger improvements.
This observation highlights the importance of semantic relatedness in
cross-domain recommendation and suggests potential directions for future work,
such as source-domain selection and weighting strategies.

\section{Summary}

In summary, the experimental results presented in this chapter demonstrate the
effectiveness of the proposed DUP-OT framework. The results confirm that
incorporating source-domain information via optimal transport significantly
improves target-domain recommendation performance, that distribution-based user
preference modeling provides clear advantages over conventional approaches, and
that DUP-OT outperforms existing cross-domain recommendation methods in
non-overlapping settings.
\chapter{Conclusion}
\label{chap:conclusion}

This thesis investigated cross-domain recommendation under non-overlapping
settings, where neither users nor items are shared across domains during the
training phase.
Such scenarios can arise in real-world applications, yet remain
challenging for conventional methods that rely on shared entities for knowledge transfer or domain adaptation.
What's more, most existing approaches represent user preferences as fixed-point
embeddings, which may fail to capture the inherent uncertainty and diversity of user interests.

To address these challenges, we proposed a novel framework, DUP-OT, which models user
preferences as probability distributions and performs cross-domain knowledge
transfer via optimal transport, without relying on overlapping users or items
during training, while allowing such information to be optionally incorporated
at inference time when available.

\section{Summary of Findings}

The core idea of DUP-OT is to represent user preferences as mixture weights over
Gaussian components in a shared latent space, rather than as fixed-point
embeddings, and to enable cross-domain knowledge transfer under a strictly
non-overlapping training setting. By fitting Gaussian Mixture Models on item
representations in each domain, user preferences can be expressed as
distributional representations that naturally capture uncertainty and
multi-modal interest structures. Optimal transport is then employed to align
preference weights across domains at the component level, enabling principled
and interpretable cross-domain transfer without relying on overlapping users or
items during training.

Extensive experiments conducted on multiple Amazon Review datasets demonstrate
the effectiveness of the proposed framework under strictly non-overlapping
training settings. The experimental results show that
DUP-OT consistently outperforms strong single-domain baselines as well as
representative cross-domain recommendation methods under non-overlapping
settings. In particular, incorporating source-domain preference information via
optimal transport leads to substantial improvements in target-domain rating
prediction accuracy. Even without leveraging source-domain data, the
distribution-based preference modeling adopted in DUP-OT yields superior
performance compared to conventional vector-based approaches, highlighting the
importance of modeling uncertainty and preference heterogeneity.

Additional analyses further indicate that the proposed framework is robust with
respect to random initialization and training dynamics. The consistent
performance across multiple random seeds suggests that the observed improvements
are not due to incidental effects, but rather reflect the inherent advantages of
the proposed modeling and transfer strategy.

\section{Contributions and Implications}

This thesis makes several contributions to the study of cross-domain
recommendation. First, it introduces a distribution-based formulation of user
preferences that departs from the dominant paradigm of point embeddings. By
modeling preferences as mixture weights over latent components, the proposed
approach provides a more expressive representation capable of capturing diverse
and uncertain user behaviors.

Second, this work highlights optimal transport as a principled and flexible
mechanism for cross-domain preference transfer under non-overlapping training
settings. Unlike approaches that rely on adversarial alignment or shared
entities during training, DUP-OT leverages optimal transport at inference time
to incorporate source-domain preference distributions when such information is
available. By performing explicit distribution-level transfer and fusion rather
than representation-level alignment, the proposed framework enables stable,
interpretable, and scenario-adaptive knowledge transfer across domains. This
perspective broadens the applicability of optimal transport in recommender
systems and underscores its potential for handling domain shift and data
sparsity.

From a practical standpoint, the proposed framework is particularly well-suited
for real-world recommendation scenarios involving cold-start users or fragmented
platform ecosystems. By avoiding reliance on shared users or items during
training and allowing auxiliary cross-domain user information to be optionally
incorporated at inference time, DUP-OT can provide meaningful recommendations
even when target-domain data are limited. This flexibility makes the framework
applicable to a wide range of industrial settings, such as cross-platform content
recommendation and emerging service deployment.

\section{Limitations}

Despite its effectiveness, the proposed approach has several limitations that
should be acknowledged. First, the structure of the Gaussian Mixture Model is
closely tied to the scale and distribution of items within each domain. When the
number of items or their distribution changes substantially, the appropriate
number of mixture components may also change, which in turn requires
reconfiguring the second-stage neural architectures. In particular, both the
multi-layer perceptrons used for modeling user-specific mixture weights and
those used for rating prediction are structurally coupled to the number of
mixture components. As a result, adapting the framework to domains with
significantly different item scales or supporting dynamic scaling may require
architectural modification, posing challenges for large-scale or rapidly
evolving deployment scenarios.

Second, the proposed framework implicitly assumes a certain degree of semantic
relatedness between the source and target domains. While the model does not rely
on overlapping users or items during training, effective cross-domain transfer
still depends on the existence of shared or compatible latent structures across
domains. When the semantic gap between domains is extremely large, learning a
meaningful shared representation space becomes challenging, which may limit the
effectiveness of distribution-level alignment and degrade the quality of the
resulting transport plan.

In addition, the experimental evaluation in this work has certain limitations.
Specifically, we focus primarily on rating prediction and do not include ranking-based metrics such as NDCG or Hit@K, which are widely used in top-$K$ recommendation settings.
Furthermore, although we compare against several representative baselines, some of them are not the most recent state-of-the-art methods.
A more comprehensive evaluation with stronger and more up-to-date baselines, together with ranking-oriented metrics, is left for future work.

Finally, the temporal dynamics of user preferences are incorporated only through
a lightweight timestamp encoding. While this design choice balances model
complexity and performance, it does not explicitly model long-term sequential
dependencies or evolving user interests over time.

\section{Future Work}

Several directions for future research can be explored to extend this work.
One promising direction is to investigate more expressive distribution families
for modeling user preferences, such as mixtures with full covariance structures
or non-Gaussian components, which may capture richer interaction patterns.

Another important direction is to integrate more advanced temporal modeling
techniques into the framework. Incorporating sequential models or continuous-time
dynamics could enable more accurate modeling of preference evolution and further
improve recommendation quality.

Another important direction is to broaden the evaluation protocol of the proposed framework.
In particular, extending the experiments to top-$K$ recommendation settings with ranking-based metrics such as NDCG and Hit@K would provide a more comprehensive assessment of practical performance.
Moreover, comparing against stronger and more recent state-of-the-art baselines on larger and more diverse benchmarks would further clarify the advantages and limitations of the proposed approach.

In addition, extending the proposed approach to multi-source or multi-target
cross-domain scenarios would enhance its applicability in complex real-world
environments.

In conclusion, this thesis demonstrates that modeling user preferences as
distributions and transferring them via optimal transport provides a powerful
and flexible solution to cross-domain recommendation under non-overlapping
training settings. The proposed DUP-OT framework offers both theoretical insights and
practical benefits, and serves as a foundation for future research in
distribution-based recommender systems.



\bibliography{reference}

\pagestyle{fancy}
\lhead{}
\rhead{}

\chapter*{Appendix}
\addcontentsline{toc}{chapter}{Appendix} % Add to table of contents

% \appendix

% 付録は chapter の 1 つとして作りますが、章番号は表示しません。
% また付録の 1 つずつはアルファベットで番号付けをするのが一般的です。
\setcounter{section}{0} % section の番号をゼロにリセットする
\renewcommand{\thesection}{\Alph{section}} % 数字ではなくアルファベットで数える
\setcounter{equation}{0} % 式番号を A.1 のようにする
\renewcommand{\theequation}{\Alph{section}.\arabic{equation}}
\setcounter{figure}{0} % 図番号
\renewcommand{\thefigure}{\Alph{section}.\arabic{figure}}
\setcounter{table}{0} % 表番号
\renewcommand{\thetable}{\Alph{section}.\arabic{table}}


\section{Graph-Enhanced EEG Foundation Models} % Appendix
\chapter*{Publications}
\addcontentsline{toc}{chapter}{Publications}

\begin{itemize}
	\item 肖 子吟 and 鈴村 豊太郎. 非重複クロスドメイン推薦に向けたガウス混合ユーザ嗜好分布と最適輸送の統合フレームワーク — DUP-OT —. WebDB夏のワークショップ2025
	\item Ziyin Xiao and Toyotaro Suzumura. Modeling User Preferences as Distributions for Optimal Transport-based Cross-domain Recommendation under Non-overlapping Settings. In the Proceedings of The 30th Pacific-Asia Conference on Knowledge Discovery and Data Mining. (Submitted)
\end{itemize} % Publication
\chapter*{Acknowledgment}
\addcontentsline{toc}{chapter}{Acknowledgement}

This work was supported by ... % Acknowledgment

\end{document}