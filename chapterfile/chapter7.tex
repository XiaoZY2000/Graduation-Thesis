\chapter{Conclusion}
\label{chap:conclusion}

This thesis investigated cross-domain recommendation under non-overlapping
settings, where neither users nor items are shared across domains during the
training phase.
Such scenarios can arise in real-world applications, yet remain
challenging for conventional methods that rely on shared entities for knowledge transfer or domain adaptation.
What's more, most existing approaches represent user preferences as fixed-point
embeddings, which may fail to capture the inherent uncertainty and diversity of user interests.

To address these challenges, we proposed a novel framework, DUP-OT, which models user
preferences as probability distributions and performs cross-domain knowledge
transfer via optimal transport, without relying on overlapping users or items
during training, while allowing such information to be optionally incorporated
at inference time when available.

\section{Summary of Findings}

The core idea of DUP-OT is to represent user preferences as mixture weights over
Gaussian components in a shared latent space, rather than as fixed-point
embeddings, and to enable cross-domain knowledge transfer under a strictly
non-overlapping training setting. By fitting Gaussian Mixture Models on item
representations in each domain, user preferences can be expressed as
distributional representations that naturally capture uncertainty and
multi-modal interest structures. Optimal transport is then employed to align
preference weights across domains at the component level, enabling principled
and interpretable cross-domain transfer without relying on overlapping users or
items during training.

Extensive experiments conducted on multiple Amazon Review datasets demonstrate
the effectiveness of the proposed framework under strictly non-overlapping
training settings. The experimental results show that
DUP-OT consistently outperforms strong single-domain baselines as well as
representative cross-domain recommendation methods under non-overlapping
settings. In particular, incorporating source-domain preference information via
optimal transport leads to substantial improvements in target-domain rating
prediction accuracy. Even without leveraging source-domain data, the
distribution-based preference modeling adopted in DUP-OT yields superior
performance compared to conventional vector-based approaches, highlighting the
importance of modeling uncertainty and preference heterogeneity.

Additional analyses further indicate that the proposed framework is robust with
respect to random initialization and training dynamics. The consistent
performance across multiple random seeds suggests that the observed improvements
are not due to incidental effects, but rather reflect the inherent advantages of
the proposed modeling and transfer strategy.

\section{Contributions and Implications}

This thesis makes several contributions to the study of cross-domain
recommendation. First, it introduces a distribution-based formulation of user
preferences that departs from the dominant paradigm of point embeddings. By
modeling preferences as mixture weights over latent components, the proposed
approach provides a more expressive representation capable of capturing diverse
and uncertain user behaviors.

Second, this work highlights optimal transport as a principled and flexible
mechanism for cross-domain preference transfer under non-overlapping training
settings. Unlike approaches that rely on adversarial alignment or shared
entities during training, DUP-OT leverages optimal transport at inference time
to incorporate source-domain preference distributions when such information is
available. By performing explicit distribution-level transfer and fusion rather
than representation-level alignment, the proposed framework enables stable,
interpretable, and scenario-adaptive knowledge transfer across domains. This
perspective broadens the applicability of optimal transport in recommender
systems and underscores its potential for handling domain shift and data
sparsity.

From a practical standpoint, the proposed framework is particularly well-suited
for real-world recommendation scenarios involving cold-start users or fragmented
platform ecosystems. By avoiding reliance on shared users or items during
training and allowing auxiliary cross-domain user information to be optionally
incorporated at inference time, DUP-OT can provide meaningful recommendations
even when target-domain data are limited. This flexibility makes the framework
applicable to a wide range of industrial settings, such as cross-platform content
recommendation and emerging service deployment.

\section{Limitations}

Despite its effectiveness, the proposed approach has several limitations that
should be acknowledged. First, the structure of the Gaussian Mixture Model is
closely tied to the scale and distribution of items within each domain. When the
number of items or their distribution changes substantially, the appropriate
number of mixture components may also change, which in turn requires
reconfiguring the second-stage neural architectures. In particular, both the
multi-layer perceptrons used for modeling user-specific mixture weights and
those used for rating prediction are structurally coupled to the number of
mixture components. As a result, adapting the framework to domains with
significantly different item scales or supporting dynamic scaling may require
architectural modification, posing challenges for large-scale or rapidly
evolving deployment scenarios.

Second, the proposed framework implicitly assumes a certain degree of semantic
relatedness between the source and target domains. While the model does not rely
on overlapping users or items during training, effective cross-domain transfer
still depends on the existence of shared or compatible latent structures across
domains. When the semantic gap between domains is extremely large, learning a
meaningful shared representation space becomes challenging, which may limit the
effectiveness of distribution-level alignment and degrade the quality of the
resulting transport plan.

In addition, the experimental evaluation in this work has certain limitations.
Specifically, we focus primarily on rating prediction and do not include ranking-based metrics such as NDCG or Hit@K, which are widely used in top-$K$ recommendation settings.
Furthermore, although we compare against several representative baselines, some of them are not the most recent state-of-the-art methods.
A more comprehensive evaluation with stronger and more up-to-date baselines, together with ranking-oriented metrics, is left for future work.

Finally, the temporal dynamics of user preferences are incorporated only through
a lightweight timestamp encoding. While this design choice balances model
complexity and performance, it does not explicitly model long-term sequential
dependencies or evolving user interests over time.

\section{Future Work}

Several directions for future research can be explored to extend this work.
One promising direction is to investigate more expressive distribution families
for modeling user preferences, such as mixtures with full covariance structures
or non-Gaussian components, which may capture richer interaction patterns.

Another important direction is to integrate more advanced temporal modeling
techniques into the framework. Incorporating sequential models or continuous-time
dynamics could enable more accurate modeling of preference evolution and further
improve recommendation quality.

Another important direction is to broaden the evaluation protocol of the proposed framework.
In particular, extending the experiments to top-$K$ recommendation settings with ranking-based metrics such as NDCG and Hit@K would provide a more comprehensive assessment of practical performance.
Moreover, comparing against stronger and more recent state-of-the-art baselines on larger and more diverse benchmarks would further clarify the advantages and limitations of the proposed approach.

In addition, extending the proposed approach to multi-source or multi-target
cross-domain scenarios would enhance its applicability in complex real-world
environments.

In conclusion, this thesis demonstrates that modeling user preferences as
distributions and transferring them via optimal transport provides a powerful
and flexible solution to cross-domain recommendation under non-overlapping
training settings. The proposed DUP-OT framework offers both theoretical insights and
practical benefits, and serves as a foundation for future research in
distribution-based recommender systems.