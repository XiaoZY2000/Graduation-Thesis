\chapter{Conclusion}
\label{chap:conclusion}

This thesis investigated the problem of cross-domain recommendation under
non-overlapping settings, where users and items are not shared across domains.
Such scenarios are common in real-world applications, yet remain challenging
for conventional recommendation methods that rely on shared entities or direct
interaction overlap. To address this challenge, we proposed a novel framework,
DUP-OT, which models user preferences as probability distributions and performs
cross-domain knowledge transfer through optimal transport.

\section{Summary of Findings}

The core idea of DUP-OT is to represent user preferences as mixture weights over
Gaussian components in a shared latent space, rather than as fixed-point
embeddings. By fitting Gaussian Mixture Models on item representations in each
domain, user preferences can be expressed as distributional representations that
naturally capture uncertainty and multi-modal interest structures. Optimal
transport is then employed to align preference weights across domains at the
component level, enabling principled and interpretable cross-domain transfer.

Extensive experiments conducted on multiple Amazon Review datasets demonstrate
the effectiveness of the proposed framework. The experimental results show that
DUP-OT consistently outperforms strong single-domain baselines as well as
representative cross-domain recommendation methods under non-overlapping
settings. In particular, incorporating source-domain preference information via
optimal transport leads to substantial improvements in target-domain rating
prediction accuracy. Even without leveraging source-domain data, the
distribution-based preference modeling adopted in DUP-OT yields superior
performance compared to conventional vector-based approaches, highlighting the
importance of modeling uncertainty and preference heterogeneity.

Additional analyses further indicate that the proposed framework is robust with
respect to random initialization and training dynamics. The consistent
performance across multiple random seeds suggests that the observed improvements
are not due to incidental effects, but rather reflect the inherent advantages of
the proposed modeling and transfer strategy.

\section{Contributions and Implications}

This thesis makes several contributions to the study of cross-domain
recommendation. First, it introduces a distribution-based formulation of user
preferences that departs from the dominant paradigm of point embeddings. By
modeling preferences as mixture weights over latent components, the proposed
approach provides a more expressive representation capable of capturing diverse
and uncertain user behaviors.

Second, this work demonstrates that optimal transport offers an effective and
flexible mechanism for cross-domain preference transfer in non-overlapping
scenarios. Unlike methods that rely on adversarial alignment or shared entities,
DUP-OT performs explicit alignment at the distribution level, resulting in more
stable and interpretable knowledge transfer. This perspective broadens the scope
of optimal transport in recommender systems and highlights its potential for
handling domain shift and data sparsity.

From a practical standpoint, the proposed framework is particularly well-suited
for real-world recommendation scenarios involving cold-start users or fragmented
platform ecosystems. By leveraging auxiliary information from related domains
without requiring shared users or items, DUP-OT can provide meaningful
recommendations even when target-domain data are limited. This characteristic
makes the framework applicable to a wide range of industrial settings, such as
cross-platform content recommendation and emerging service deployment.

\section{Limitations}

Despite its effectiveness, the proposed approach has several limitations that
should be acknowledged. First, the use of Gaussian Mixture Models introduces
additional computational overhead compared to simple embedding-based methods,
particularly when the number of mixture components is large. Although diagonal
covariance matrices and dimensionality reduction are employed to improve
efficiency, scalability to extremely large item sets remains a concern.

Second, the current framework assumes that a shared latent representation space
can be learned across domains through a common encoder. When the semantic gap
between domains is extremely large, this assumption may not hold, potentially
limiting the effectiveness of cross-domain alignment. In such cases, the quality
of the learned transport plan may degrade.

Finally, the temporal dynamics of user preferences are incorporated only through
a lightweight timestamp encoding. While this design choice balances model
complexity and performance, it does not explicitly model long-term sequential
dependencies or evolving user interests over time.

\section{Future Work}

Several directions for future research can be explored to extend this work.
One promising direction is to investigate more expressive distribution families
for modeling user preferences, such as mixtures with full covariance structures
or non-Gaussian components, which may capture richer interaction patterns.

Another important direction is to integrate more advanced temporal modeling
techniques into the framework. Incorporating sequential models or continuous-time
dynamics could enable more accurate modeling of preference evolution and further
improve recommendation quality.

In addition, extending the proposed approach to multi-source or multi-target
cross-domain scenarios would enhance its applicability in complex real-world
environments. Finally, exploring more efficient optimal transport solvers and
approximation techniques could further improve scalability and enable deployment
on large-scale industrial datasets.

In conclusion, this thesis demonstrates that modeling user preferences as
distributions and transferring them via optimal transport provides a powerful
and flexible solution to cross-domain recommendation under non-overlapping
settings. The proposed DUP-OT framework offers both theoretical insights and
practical benefits, and serves as a foundation for future research in
distribution-based recommender systems.