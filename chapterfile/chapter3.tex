\chapter{Related Work} \label{chap:related_work}

\section{Recommender Systems}
% The related work of recommender systems can be broadly categorized into several areas based on the techniques and approaches used. This section provides an overview of the key developments in each of these areas.

\subsection{Divide by Technique}
\subsubsection{Traditional Recommender Systems}
% Include collaborative filtering, content-based filtering, and hybrid methods.
% Without using Machine Learning or Deep Learning techniques.
Collaborative filtering (CF) is one of the foundational approaches in recommender systems, operating on the principle that users with similar preferences will like similar items\cite{BOBADILLA2013109}. User-based CF finds neighbors with comparable rating patterns, while item-based CF identifies items that are similar based on user interactions. Content-based filtering recommends items by matching item features with user preferences learned from historical data. Hybrid methods combine multiple approaches to leverage their complementary strengths, addressing individual limitations such as sparsity and cold-start problems.

These traditional methods have proven effective for many applications but face challenges including data sparsity, scalability issues, and difficulty in capturing complex non-linear patterns in user behavior.

\subsubsection{Machine Learning-based Recommender Systems}
% Include some classic machine learning techniques, like XGBoost, Random Forest, SVM, Matrix Factorization, etc.
Machine learning techniques have been widely adopted in recommender systems to enhance prediction accuracy and address limitations of traditional methods. Matrix factorization (MF) techniques, such as Singular Value Decomposition (SVD), decompose the user-item interaction matrix into latent factors, enabling the capture of underlying user preferences and item characteristics. Neighborhood-based methods have also been extended using machine learning algorithms to improve similarity measures and prediction models.

\subsubsection{Deep Learning-based Recommender Systems}
Deep learning has revolutionized recommender systems by enabling the modeling of complex user-item interactions and capturing non-linear relationships. Neural Collaborative Filtering (NCF) employs multi-layer perceptrons to learn user-item interaction functions, outperforming traditional MF methods. Convolutional Neural Networks (CNNs) have been utilized to extract features from item content, such as images and text, enhancing recommendation quality. Recurrent Neural Networks (RNNs) have been applied to model sequential user behavior, capturing temporal dynamics in user preferences.

\subsection{Divide by Application Domain}

\subsubsection{Sequential Recommender Systems}

\subsubsection{Graph-based Recommender Systems}

\subsubsection{Cross-Domain Recommender Systems}

\subsubsection{Generative Recommender Systems}