\chapter{Experiments} \label{chap:experiments}

This chapter presents a comprehensive experimental evaluation of the proposed
\textbf{DUP-OT} framework. The primary goal of these experiments is to examine
whether modeling user preferences as probability distributions and transferring
such distributions across domains via optimal transport can effectively improve
recommendation performance in non-overlapping cross-domain scenarios.

Unlike conventional conference-style experimental sections that primarily focus
on numerical comparisons, this chapter emphasizes experimental rigor,
reproducibility, and interpretability. We therefore provide detailed descriptions
of dataset construction, temporal splitting strategies, architectural design
choices, training protocols, and baseline configurations. These details are
essential for understanding the behavior of the proposed method and for enabling
future work to reproduce and extend our results.

\section{Research Questions}

The experiments are designed to answer the following research questions:

\begin{itemize}
	\item \textbf{RQ1.} Does incorporating source-domain information improve
	      recommendation performance in the target domain under non-overlapping user
	      and item settings?

	\item \textbf{RQ2.} Does representing user preferences as probability
	      distributions provide advantages over conventional vector-based
	      representations in single-domain recommendation?

	\item \textbf{RQ3.} How does the proposed DUP-OT framework compare with
	      existing cross-domain recommendation methods that are designed for
	      non-overlapping scenarios?
\end{itemize}

Each research question is addressed through carefully designed experiments and
controlled comparisons, including ablation studies and cross-model evaluations.

\section{Datasets}

We conduct experiments on the Amazon Review 5-core datasets~\cite{J_Ni_2019},
which have been widely adopted in recommender system research due to their scale,
diversity, and availability of both interaction data and rich textual reviews.

Specifically, we use four domains:
\textit{Digital Music}, \textit{Movies and TV}, \textit{Video Games}, and
\textit{Electronics}. Among them, Electronics is treated as the target domain,
while the remaining three domains are used as source domains in different
source--target configurations.

The Amazon Review datasets contain explicit ratings ranging from 1 to 5, which
makes them suitable for evaluating rating prediction performance. Moreover, the
availability of user- and item-level textual reviews enables the construction of
semantically meaningful representations beyond purely interaction-based models.

\subsection{5-core Filtering}

To reduce extreme sparsity and ensure a minimum level of interaction density, we
apply standard 5-core filtering to each domain independently. After filtering,
each user and item has at least five interactions within the corresponding
domain. This preprocessing step is consistent with prior work and allows for
fair comparison with baseline models.

\subsection{Temporal Splitting Strategy}

In the target domain (Electronics), interactions are split chronologically into
training, validation, and test sets with a ratio of 8:1:1. Chronological splitting
is adopted to simulate realistic recommendation scenarios in which models are
trained on past data and evaluated on future interactions.

To prevent temporal information leakage across domains, we further filter the
source-domain datasets by removing interactions that occur later than the
earliest validation timestamp in the target domain. This ensures that all
source-domain information used during training is temporally prior to the target
domain evaluation period.

This temporal alignment strategy is particularly important for cross-domain
recommendation, as naive inclusion of future source-domain interactions could
lead to overly optimistic performance estimates.

\section{Overview of DUP-OT}

The DUP-OT framework consists of three major stages: unified representation
learning, distribution-based preference modeling, and cross-domain preference
transfer. Each stage is designed to address a specific limitation of existing
cross-domain recommendation approaches.

\subsection{Unified Representation Learning}

In the first stage, we employ a shared pre-trained sentence encoder,
\texttt{all-MiniLM-L6-v2}, from the Sentence-Transformers library to encode user
and item reviews into dense semantic embeddings. The encoder is kept frozen
throughout training to preserve its general semantic knowledge and to reduce
computational cost.

A shared autoencoder is then trained on embeddings from all involved domains to
project them into a unified low-dimensional latent space. By sharing the
autoencoder across domains, we encourage the model to learn domain-invariant
semantic representations while still preserving domain-specific information.

\subsection{Distribution-Based User Preference Modeling}

In the second stage, we model user preferences as probability distributions
rather than fixed-point embeddings. For each domain, we fit an item-level Gaussian
Mixture Model (GMM) on the learned item embeddings. The use of GMMs allows the
model to capture multi-modal structure in the item space.

The number of mixture components is automatically determined using the
\texttt{BayesianGaussianMixture} estimator, which relies on variational Bayesian
inference to avoid manual hyperparameter tuning. This design choice improves
robustness across domains with different data scales.

For each user, a lightweight MLP predicts mixture weights over the shared GMM
components. These weights represent the user’s preference distribution over
latent semantic clusters. A separate MLP is trained to predict ratings based on
the interaction between user preference distributions and item embeddings.

\subsection{Cross-Domain Preference Transfer}

In the third stage, we perform cross-domain preference transfer using Optimal
Transport (OT). OT provides a principled way to align probability distributions
across domains by minimizing the cost of transporting mass between distributions.

By aligning user preference distributions rather than point embeddings, DUP-OT
achieves flexible and interpretable cross-domain knowledge transfer. During
inference, transferred source-domain preferences are fused with target-domain
preferences to produce final rating predictions.

\section{Training Protocol}

All models are trained using the Adam optimizer with a learning rate of
$3\times10^{-4}$ and a batch size of 256. Early stopping is applied based on
validation loss in the target domain to prevent overfitting.

The sentence encoder is frozen, while the shared autoencoder is trained jointly
across domains. Domain-specific GMMs and MLPs are trained independently. Users
that appear only in validation or test sets are treated as cold-start users.

All experiments are implemented in PyTorch and conducted on a single NVIDIA GPU.
To reduce variance caused by random initialization, each experiment is repeated
with five random seeds $\{7, 11, 17, 33, 2025\}$, and all reported results are
averaged across runs.

\section{Baseline Models}

We compare DUP-OT with both single-domain and cross-domain baselines.

\textbf{Single-domain baselines} include LightGCN~\cite{He_Deng_Wang_Li_Zhang_Wang_2020_LightGCN} and
NeuMF~\cite{He_Liao_Zhang_Nie_Hu_Chua_2017_NCF}, which are representative collaborative
filtering models. These models are implemented using the RecBole framework and
trained solely on the target domain.

\textbf{Cross-domain baseline} includes TDAR~\cite{Yu_Lin_Ge_Ou_Qin_2020_TDAR}, a
text-enhanced domain adaptation method designed for non-overlapping scenarios.
We adapt TDAR to the rating prediction setting by replacing its original
contrastive loss with a regression objective.

\section{Evaluation Metrics}

All models are evaluated on rating prediction in the target domain. We report
Root Mean Square Error (RMSE) and Mean Absolute Error (MAE), which are standard
metrics for explicit-feedback recommendation tasks. Lower values indicate better
performance.
