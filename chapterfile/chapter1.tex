\chapter{Introduction}
\label{chap:introduction}

\section{Background}

Information overload has become a critical challenge for users on modern online
platforms, including e-commerce websites, streaming services, and social
media~\cite{Bawden_Robinson_2020_info_overload,Lv_Liu_2022_info_overload,Zhang_Cao_Liu_2023_info_overload}.
Recommender systems play a crucial role in alleviating this problem by providing
personalized guidance and selecting items with the highest predicted relevance
from a large pool of candidates~\cite{Resnick_Varian_1997_recommender,Burke_2002_hybrid_recommender,Burke_Felfernig_Göker_2011_recommender_overview}.
By analyzing historical user--item interactions, recommender systems enable
users to efficiently discover relevant content, products, or services, thereby
improving user satisfaction and engagement.

Beyond enhancing user experience, recommender systems also generate substantial
business value by increasing sales, customer retention, and overall platform
activity~\cite{Jannach_Jugovac_2019_business_value_recommender}.
As the scale, diversity, and dynamics of online content continue to grow, the
importance of effective and robust recommender systems is increasingly
recognized in both academia and industry.

Despite their success, recommender systems face several fundamental challenges
that limit their performance.
One major challenge is data sparsity, where users interact with only a small
fraction of available items, making it difficult to accurately infer user
preferences.
This issue is particularly severe in emerging platforms or newly established
services with limited historical data.
Another critical challenge is the cold-start problem, which arises when new
users or items enter the system and insufficient interaction history is
available to support reliable recommendations.
These challenges arise from the limited availability of user--item interaction
data and can significantly degrade recommendation quality.

To address these challenges, cross-domain recommendation has emerged as a
promising paradigm that leverages information from a related source domain to
improve recommendation performance in a target domain.
By transferring knowledge across domains, cross-domain recommendation can
alleviate data sparsity and cold-start issues, particularly in data-scarce
settings.
However, most existing cross-domain recommendation methods rely on the
assumption that users or items overlap between the source and target domains,
and such overlapping entities are used as bridges for knowledge transfer during
training.
In many real-world scenarios, this assumption does not hold, as domains may be
entirely disjoint due to privacy constraints, system isolation, or independent
platform design. What's more, even when some overlapping users exist, they often
constitute only a small fraction of the overall user base, limiting their
effectiveness for knowledge transfer and increasing the risk of overfitting to
these few entities.

Moreover, traditional recommender systems typically represent user preferences
as point estimates in a latent space, such as embedding vectors.
Such representations are widely used and can be effective in many scenarios.
However, point embeddings may be often insufficient to capture the uncertainty
and multi-faceted nature of real user behavior, especially under
sparse data conditions.
This limitation further motivates the exploration of more expressive preference
representations for cross-domain recommendation.

Motivated by these challenges, this thesis proposes a novel cross-domain
recommendation framework that models user preferences as probabilistic distributions,
specifically, Gaussian Mixture Models (GMMs), and integrates optimal transport techniques to enable knowledge transfer
under non-overlapping training settings.
By representing user preferences as probability distributions rather than point
estimates, the proposed approach captures both the diversity and uncertainty of
user interests. And the multi-peaked structure of GMMs allows for modeling
multiple distinct user preferences effectively.
Furthermore, optimal transport provides a principled mechanism for aligning
user preference representations across domains, facilitating effective
knowledge transfer without relying on overlapping users or items during the
training phase, thereby mitigating the risk of overfitting caused
by scarce overlapping entities.

The proposed framework consists of two key components:
(1) modeling user preferences as Gaussian mixture distributions to capture the
complex structure of user behavior within each domain, and
(2) employing optimal transport to align these preference representations
between the source and target domains.
Together, these components enable robust cross-domain recommendation and improve
recommendation accuracy in the target domain under strictly non-overlapping
training settings.

\paragraph{Training-time non-overlap and inference-time overlap.}

It is important to clarify our assumptions regarding cross-domain user
information. In this work, we explicitly distinguish between its availability
during training and inference.
Specifically, we assume a \emph{non-overlapping setting during training}, where
no explicit user or item correspondences between domains are available.
This design choice helps avoid the risk of overfitting to scarce overlapping entities,
and also reflects realistic industrial scenarios in which overlapping users or items
across different domains are typically very sparse (e.g., across different categories on Amazon).

At inference time, however, the situation can be different.
Since model parameters are fixed during inference,
incorporating additional cross-domain user correspondence does not introduce the risk of training-time overfitting.
Therefore, when such correspondence is available, the proposed framework can exploit overlapping user information at inference time to further enhance recommendation performance.

Overall, the proposed DUP-OT framework is designed to operate under non-overlapping settings by
modeling user preferences as structured distributions to enhance expressive capacity,
while achieving cross-domain knowledge transfer through optimal transport based on semantic alignment.
When overlapping user information becomes available at inference time,
it can be incorporated as an optional enhancement rather than a prerequisite,
allowing the model to flexibly adapt to different real-world scenarios.

\section{Contributions}

The main contributions of this thesis are summarized as follows:
\begin{itemize}
	\item We propose a novel cross-domain recommendation framework designed for
	      non-overlapping training settings, where no shared users or items are
	      available between domains.
	\item We model user preferences as Gaussian Mixture Models (GMMs), enabling a
	      distribution-based representation that captures uncertainty and
	      multi-interest user preferences more effectively than conventional
	      point-embedding approaches.
	\item We integrate optimal transport techniques to align user preference
	      representations across domains, enabling principled and interpretable
	      cross-domain knowledge transfer without requiring overlapping entities
	      during training.
	\item We conduct extensive experiments on multiple Amazon Review datasets to
	      evaluate the proposed framework under various cross-domain settings.
	      The results demonstrate that the proposed method consistently
	      outperforms both traditional single-domain recommenders and
	      representative cross-domain baselines.
	\item We perform ablation studies to analyze the effects of distribution-based
	      preference modeling and cross-domain knowledge transfer, providing
	      insights into the contributions of each component of the framework.
\end{itemize}

\section{Thesis Organization}

The remainder of this thesis is organized as follows:
\begin{itemize}
	\item Chapter~\ref{chap:preliminaries} introduces the fundamental concepts and
	      techniques relevant to this work, including recommender systems,
	      cross-domain recommendation, Gaussian Mixture Models, and optimal
	      transport.
	\item Chapter~\ref{chap:related_work} reviews related work in conventional recommender
	      systems, cross-domain recommendation, distributional user preference
	      modeling, and optimal transport for representation alignment.
	\item Chapter~\ref{chap:proposed_method} presents the proposed DUP-OT
	      framework in detail, including shared preprocessing stage, user GMM weights
	      learning stage, and cross-domain rating prediction stage.
	\item Chapter~\ref{chap:experiments} describes the experimental setup,
	      datasets, baseline methods, and evaluation metrics used to assess the
	      performance of the proposed framework. Also, implementation details
	      are provided to facilitate reproducibility.
	\item Chapter~\ref{chap:results} reports and analyzes the experimental
	      results, including comparisons with baseline methods and ablation
	      studies.
	\item Chapter~\ref{chap:conclusion} concludes the thesis and discusses
	      limitations and directions for future research.
\end{itemize}
