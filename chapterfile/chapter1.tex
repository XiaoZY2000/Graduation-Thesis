\chapter{Introduction}\label{chap:introduction}

\section{Background}

Information overload has become a critical challenge for users on modern online
platforms, including e-commerce websites, streaming services, and social
media~\cite{Bawden_Robinson_2020_info_overload,Lv_Liu_2022_info_overload,Zhang_Cao_Liu_2023_info_overload}.
Recommender systems play a crucial role in alleviating this problem by providing
personalized guidance, selecting items with the highest predicted utility from a
large pool of candidates~\cite{Resnick_Varian_1997_recommender,Burke_2002_hybrid_recommender,Burke_Felfernig_Göker_2011_recommender_overview}.
By analyzing historical user-item interactions, recommender systems help users
discover relevant content, products, or services, thereby improving user
satisfaction and engagement.
Beyond enhancing user experience, recommender systems also generate substantial
business value by increasing sales, customer retention, and overall platform
activity~\cite{Jannach_Jugovac_2019_business_value_recommender}.
As the scale and diversity of online content continue to grow, the importance of
effective recommender systems becomes increasingly pronounced.

Despite their success, recommender systems face several fundamental challenges
that limit their performance.
One major challenge is data sparsity, where users interact with only a small
fraction of available items, making it difficult to accurately infer user
preferences.
This issue is particularly severe in emerging or small-scale platforms with limited
interaction data.
Another critical challenge is the cold-start problem, which arises when new users
or items enter the system and insufficient historical data is available to
support reliable recommendations.
Both data sparsity and cold-start problems arise from the limited availability of
user-item interaction data and can significantly degrade recommendation quality.

To address these challenges, cross-domain recommendation has emerged as a
promising paradigm that leverages information from a related source domain to
improve recommendation performance in a target domain.
By transferring knowledge across domains, cross-domain recommendation is able to
alleviate data sparsity and cold-start issues, particularly in data-scarce
settings.
However, most existing cross-domain recommendation methods rely on the
assumption that users or items overlap between the source and target domains, and
such overlaps are used as bridges for knowledge transfer during the training
stage.
In many real-world scenarios, however, this assumption does not hold, as domains
may be entirely disjoint due to privacy constraints, system isolation, or
independent platform design.
Moreover, traditional recommender systems typically represent user preferences as
point estimates in a latent space, which are often insufficient to capture the
uncertainty and multi-faceted nature of real user behavior.

Motivated by these limitations, this thesis proposes a novel cross-domain
recommendation framework that models user preferences as Gaussian Mixture Models
(GMMs) and integrates optimal transport techniques to enable knowledge transfer
under non-overlapping training settings.
By representing user preferences as probability distributions rather than point
estimates, the proposed approach captures both the diversity and uncertainty of
user interests.
Furthermore, optimal transport provides a principled mechanism for aligning user
preference distributions across domains, facilitating effective knowledge
transfer without relying on overlapping users or items during the training
phase.

The proposed framework consists of two key components:
(1) modeling user preferences as Gaussian mixture distributions to capture the
complex structure of user behavior within each domain, and
(2) employing optimal transport to align these distributions between the source
and target domains.
Together, these components enable robust cross-domain recommendation and improve
recommendation accuracy in the target domain under strictly non-overlapping
training settings.

\section{Contributions}
The main contributions of this thesis are as follows:
\begin{itemize}
	\item We introduce a novel framework for cross-domain recommendation under non-overlapping settings by modeling
	\item We propose representing user preferences as Gaussian Mixture Models (GMMs) to capture the uncertainty and diversity of user behavior more effectively.
	\item We integrate optimal transport techniques to align user preference distributions across domains, enabling knowledge transfer
	\item We conduct extensive experiments on Amazon datasets across multiple domains to evaluate the performance of our proposed method. The results demonstrate that our approach significantly outperforms existing cross-domain recommendation techniques and traditional single-domain methods in terms of recommendation accuracy.
	\item We perform ablation studies to assess the effectiveness of representing user preferences as distributions and the impact of boosting target domain recommendation performance using source domain information.
	\item We provide insights into the benefits of modeling user preferences as distributions and leveraging optimal transport for cross-domain knowledge transfer, highlighting the potential of our approach for addressing challenges in recommender systems.
\end{itemize}

\section{Thesis Organization}
The remainder of this thesis is organized as follows:
\begin{itemize}
	\item In Chapter~\ref{chap:preliminaries}, we provide an overview of the fundamental concepts and techniques relevant to this research, including recommender systems, cross-domain recommendation, Gaussian Mixture Models, and optimal transport.
	\item In Chapter~\ref{chap:related_work}, we review related work on recommender systems, cross-domain recommendation, and optimal transport techniques.
	\item In Chapter~\ref{chap:proposed_method}, we present our proposed framework for cross-domain recommendation under non-overlapping settings, detailing the modeling of user preferences as GMMs and the integration of optimal transport for knowledge transfer.
	\item In Chapter~\ref{chap:experiments}, we describe the experimental setup, datasets, and evaluation metrics used to assess the performance of our proposed method.
	\item In Chapter~\ref{chap:results}, we present and discuss the experimental results, including comparisons with existing methods and ablation studies.
	\item Finally, in Chapter~\ref{chap:conclusion}, we summarize the main findings of this thesis and discuss potential directions for future research.
\end{itemize}