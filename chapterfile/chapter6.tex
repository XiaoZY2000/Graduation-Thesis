\chapter{Results}
\label{chap:results}

In this chapter, we present and analyze the experimental results obtained from
the settings described in Chapter~\ref{chap:experiments}. The primary objective
of this chapter is to empirically evaluate the effectiveness of the proposed
DUP-OT framework under the non-overlapping cross-domain recommendation setting.
We systematically compare DUP-OT with representative single-domain and
cross-domain baseline methods, and analyze the results with respect to the
research questions defined earlier. Through this analysis, we aim to provide
clear evidence of the advantages brought by cross-domain information transfer
and distribution-based user preference modeling.

\section{Overall Performance}

The overall results of the three groups of experiments introduced in
Chapter~\ref{chap:experiments} are summarized in Table~\ref{tab:results}. The
table reports rating prediction performance on the target domain in terms of
RMSE and MAE, where lower values indicate better recommendation accuracy.
Boldfaced numbers denote the best-performing method under each experimental
setting. To reduce the influence of randomness caused by parameter
initialization and data shuffling, all results shown in the table are averaged
over five independent runs with different random seeds.
The complete experimental results of the proposed DUP-OT framework under
different random seeds, including both the full model and its variants across
multiple domain pairs, are reported in Appendix~A for completeness and
reproducibility.

\begin{table*}[t]
	\centering
	\caption{Rating prediction performance of baseline models and the proposed DUP-OT framework.}
	\label{tab:results}

	\begin{subtable}{\textwidth}
		\centering
		\caption{Cross-domain recommendation results with Electronics as the target domain}
		\hspace*{-17mm}
		\begin{tabular}{l cc cc cc}
			\toprule
			$D_s \to D_t$
			      & \multicolumn{2}{c}{Digital Music $\to$ Electronics}
			      & \multicolumn{2}{c}{Movies \& TV $\to$ Electronics}
			      & \multicolumn{2}{c}{Video Games $\to$ Electronics}                                               \\
			\cmidrule(lr){2-3}
			\cmidrule(lr){4-5}
			\cmidrule(lr){6-7}
			Model & RMSE                                                & MAE             & RMSE & MAE & RMSE & MAE \\
			\midrule
			TDAR
			      & 1.5688                                              & 1.0904
			      & 1.5677                                              & 1.0896
			      & 1.5688                                              & 1.0902                                    \\

			DUP-OT (w/o source)
			      & 1.3699                                              & 0.9463
			      & 1.3494                                              & 0.8806
			      & 1.3716                                              & \textbf{0.9481}                           \\

			DUP-OT (w/ source)
			      & \textbf{1.2919}                                     & \textbf{0.8965}
			      & \textbf{1.2907}                                     & \textbf{0.8774}
			      & \textbf{1.3376}                                     & 1.0032                                    \\
			\bottomrule
		\end{tabular}
	\end{subtable}

	\vspace{1em}

	\begin{subtable}{\textwidth}
		\centering
		\caption{Single-domain recommendation results on the Electronics dataset}
		\begin{tabular}{l cc}
			\toprule
			Model    & RMSE            & MAE             \\
			\midrule
			LightGCN & 1.5317          & 1.1179          \\
			NeuMF    & 1.4599          & 1.3297          \\
			\midrule
			DUP-OT (w/o source, average)
			         & \textbf{1.3636} & \textbf{0.9250} \\
			\bottomrule
		\end{tabular}
	\end{subtable}

\end{table*}

From Table~\ref{tab:results}, it can be observed that the complete version of the
proposed DUP-OT framework consistently achieves the best performance across all
cross-domain experimental settings. This performance advantage is observed
consistently across different source domains, indicating that the proposed
framework is not sensitive to a particular domain pair. These results suggest
that incorporating source-domain knowledge via optimal transport provides
substantial benefits for improving rating prediction accuracy in the target
domain.

\section{Effect of Cross-Domain Information (RQ1)}

To investigate whether leveraging cross-domain information contributes to
improved recommendation performance, we compare DUP-OT with source-domain
information against its ablated variant trained without any source-domain
interactions. The two models share identical network architectures and training
protocols, and the only difference lies in whether source-domain preference
weights are transferred to the target domain during inference.

As shown in Table~\ref{tab:results}, DUP-OT with source-domain information
consistently outperforms DUP-OT (w/o source) across all evaluated domain pairs.
Both RMSE and MAE are reduced when source-domain information is introduced,
indicating more accurate rating prediction. This performance gap demonstrates
that source-domain user preferences provide complementary information that cannot
be fully captured by target-domain data alone. The results therefore validate the
effectiveness of cross-domain preference transfer in the proposed DUP-OT
framework.

\section{Effect of Distributional User Preference Modeling (RQ2)}

To assess the impact of modeling user preferences as distributions rather than
point embeddings, we compare DUP-OT (w/o source) with strong single-domain
baselines, namely LightGCN and NeuMF. These baseline methods represent each user
using a single latent vector and do not explicitly model uncertainty or
multi-modal preference structures.

As reported in Table~\ref{tab:results}, DUP-OT (w/o source) consistently
outperforms both LightGCN and NeuMF on the Electronics dataset, even though it does
not leverage any source-domain data. This observation indicates that
distribution-based user preference modeling provides a more expressive and
flexible representation of user interests. By modeling user preferences as
mixture weights over multiple latent components, DUP-OT is able to capture
heterogeneous and uncertain user behaviors that are difficult to represent using
single-vector embeddings.

\section{Comparison with Cross-Domain Baselines (RQ3)}

We further compare the proposed DUP-OT framework with TDAR, a representative
cross-domain recommendation method designed for non-overlapping settings. As
shown in Table~\ref{tab:results}, DUP-OT consistently achieves lower RMSE and MAE
than TDAR across all evaluated domain pairs.

These results suggest that aligning user preferences at the distribution level
via optimal transport is more effective than adversarial feature alignment
approaches for cross-domain recommendation. In particular, DUP-OT directly
operates on preference representations and performs explicit distribution
alignment, which leads to more stable and accurate knowledge transfer between
domains without relying on shared users or items.

\section{Robustness Analysis}

To evaluate the robustness of the proposed DUP-OT framework, we conduct multiple
independent runs of each experiment using different random seeds. The detailed
results for each individual run are reported in Appendix~A.
Across all runs, DUP-OT exhibits consistently strong performance, with relatively
small variations in RMSE and MAE across different initializations.

This consistency indicates that the proposed framework is stable with respect to
random initialization and training dynamics. The robustness of DUP-OT further
supports its practical applicability for cross-domain recommendation scenarios,
where reliable performance across repeated runs is an important requirement.

\section{Summary}

In summary, the experimental results presented in this chapter provide strong and
consistent evidence for the effectiveness of the proposed DUP-OT framework. The
results confirm that cross-domain preference transfer significantly improves
target-domain recommendation performance, that modeling user preferences as
distributions offers clear advantages over conventional vector-based
representations, and that DUP-OT outperforms existing cross-domain recommendation
methods under non-overlapping settings. Together, these findings demonstrate the
practical value and robustness of the proposed approach.