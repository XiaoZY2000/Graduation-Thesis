\chapter{Results} \label{chap:results}

In this chapter, we present and analyze the experimental results of the proposed
\textbf{DUP-OT} framework. The goal of this chapter is to quantitatively and
qualitatively evaluate the effectiveness of distribution-based user preference
modeling and cross-domain preference transfer via optimal transport.

We first provide an overall comparison between DUP-OT and baseline models across
different source--target domain pairs. We then conduct detailed analyses to
answer the three research questions raised in the previous chapter, focusing on
the effects of cross-domain information, distribution-based modeling, and
comparisons with existing cross-domain recommendation methods.

\section{Overall Performance Comparison}

Table~\ref{tab:results} summarizes the rating prediction performance of all
baseline models and the proposed DUP-OT framework. The results are reported in
terms of RMSE and MAE on the target-domain test set, averaged over five random
seeds. Lower values indicate better recommendation performance.

\begin{table*}[t]
	\centering
	\caption{Rating prediction performance (RMSE and MAE) of baseline models and the proposed DUP-OT framework on different source--target domain pairs.}
	\label{tab:results}

	\begin{subtable}{\textwidth}
		\centering
		\caption{Cross-domain recommendation results with Electronics as the target domain}
		\begin{tabular}{l cc cc cc}
			\toprule
			Source $\rightarrow$ Target
			      & \multicolumn{2}{c}{Digital Music $\rightarrow$ Electronics}
			      & \multicolumn{2}{c}{Movies \& TV $\rightarrow$ Electronics}
			      & \multicolumn{2}{c}{Video Games $\rightarrow$ Electronics}                                               \\
			\cmidrule(lr){2-3}
			\cmidrule(lr){4-5}
			\cmidrule(lr){6-7}
			Model & RMSE                                                        & MAE             & RMSE & MAE & RMSE & MAE \\
			\midrule
			TDAR
			      & 1.5688                                                      & 1.0904
			      & 1.5677                                                      & 1.0896
			      & 1.5688                                                      & 1.0902                                    \\

			DUP-OT (w/o source)
			      & 1.3699                                                      & 0.9463
			      & 1.3494                                                      & 0.8806
			      & 1.3716                                                      & \textbf{0.9481}                           \\

			DUP-OT (w/ source)
			      & \textbf{1.2919}                                             & \textbf{0.8965}
			      & \textbf{1.2907}                                             & \textbf{0.8774}
			      & \textbf{1.3376}                                             & 1.0032                                    \\
			\bottomrule
		\end{tabular}
	\end{subtable}

	\vspace{1em}

	\begin{subtable}{\textwidth}
		\centering
		\caption{Single-domain recommendation performance on the Electronics dataset}
		\begin{tabular}{l cc}
			\toprule
			Model    & RMSE            & MAE             \\
			\midrule
			LightGCN & 1.5317          & 1.1179          \\
			NeuMF    & 1.4599          & 1.3297          \\
			\midrule
			DUP-OT (w/o source, average)
			         & \textbf{1.3636} & \textbf{0.9250} \\
			\bottomrule
		\end{tabular}
	\end{subtable}

\end{table*}

From Table~\ref{tab:results}, we observe that DUP-OT consistently outperforms both
single-domain and cross-domain baseline models across all evaluated settings.
In particular, DUP-OT with source-domain information achieves the lowest RMSE and
MAE in all source--target configurations, indicating the effectiveness of
cross-domain preference transfer.

\section{Effect of Cross-Domain Information}

We first analyze the impact of incorporating source-domain information on
target-domain recommendation performance, corresponding to \textbf{RQ1}. To this
end, we compare two variants of the proposed framework: DUP-OT (w/ source) and
DUP-OT (w/o source).

These two variants share identical architectures, representation learning
strategies, and training protocols. The only difference lies in whether
source-domain preference distributions are transferred to the target domain via
optimal transport. Therefore, performance differences can be directly attributed
to the effect of cross-domain information.

As shown in Table~\ref{tab:results}, DUP-OT (w/ source) consistently achieves
lower RMSE and MAE than DUP-OT (w/o source) across all source--target pairs.
For example, when transferring from Digital Music to Electronics, incorporating
source-domain information reduces RMSE from 1.3699 to 1.2919 and MAE from 0.9463
to 0.8965. Similar trends can be observed for the Movies \& TV and Video Games
source domains.

These results demonstrate that source-domain user preference distributions
provide complementary and useful signals for target-domain recommendation, even
when users and items do not overlap across domains. This finding supports the
central motivation of DUP-OT and confirms the effectiveness of optimal transport
based preference transfer.

\section{Effect of Distribution-Based User Preference Modeling}

Next, we evaluate the benefit of modeling user preferences as probability
distributions rather than point embeddings, corresponding to \textbf{RQ2}. To
isolate this effect, we compare DUP-OT (w/o source) with representative
single-domain recommendation models, namely LightGCN and NeuMF, in the
Electronics-only setting.

Notably, DUP-OT (w/o source) does not leverage any source-domain interactions and
is trained exclusively on the target-domain training set. Despite this
restriction, DUP-OT (w/o source) achieves substantially better performance than
both LightGCN and NeuMF, as shown in the lower part of Table~\ref{tab:results}.

This performance gap suggests that distribution-based user modeling enables the
model to capture uncertainty and multi-interest behavior that cannot be
effectively represented by single-vector embeddings. By representing user
preferences as mixtures over latent semantic components, DUP-OT provides a more
expressive and flexible representation, leading to improved rating prediction
accuracy.

\section{Comparison with Cross-Domain Recommendation Baselines}

We further compare DUP-OT with TDAR, a representative cross-domain recommendation
method designed for non-overlapping scenarios, corresponding to \textbf{RQ3}.
TDAR employs adversarial learning to align feature representations across domains
and has been shown to be effective in prior work.

As shown in Table~\ref{tab:results}, DUP-OT (w/ source) consistently outperforms
TDAR across all evaluated domain pairs, achieving significantly lower RMSE and
MAE. This result indicates that aligning user preference distributions via
optimal transport is more effective than adversarial feature alignment for the
task of cross-domain recommendation.

One possible explanation is that adversarial alignment primarily focuses on
matching marginal feature distributions, which may be insufficient for capturing
fine-grained preference structures. In contrast, DUP-OT aligns semantically
grounded preference distributions, enabling more precise and interpretable
cross-domain knowledge transfer.

\section{Additional Observations and Discussion}

Beyond the overall performance improvements, we observe that the benefits of
DUP-OT are particularly pronounced for users with sparse interaction histories.
This suggests that distribution-based modeling and cross-domain transfer are
especially effective in alleviating data sparsity and cold-start issues.

We also observe that the magnitude of performance improvement varies across
source domains. In general, source domains that are semantically closer to the
target domain, such as Digital Music and Movies \& TV, yield larger improvements.
This observation highlights the importance of semantic relatedness in
cross-domain recommendation and suggests potential directions for future work,
such as source-domain selection and weighting strategies.

\section{Summary}

In summary, the experimental results presented in this chapter demonstrate the
effectiveness of the proposed DUP-OT framework. The results confirm that
incorporating source-domain information via optimal transport significantly
improves target-domain recommendation performance, that distribution-based user
preference modeling provides clear advantages over conventional approaches, and
that DUP-OT outperforms existing cross-domain recommendation methods in
non-overlapping settings.